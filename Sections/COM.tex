
\section{COM, DCOM e .NET}
Nell'architettura Microsoft, la GUI è sempre stata fondamentale (la finestra non è solo un elemento graco, ma è tutto
ciò a cui è costruito attorno il sistema; l'interazione fra i processi è guidata dalle finestre!). E da loro che è nata
l'idea
di dinamyc library, cioè di una libreria che si potesse caricare solo su bisogno:
idea molto dinamica, che però richiede il costo del caricamento.
Ma quindi, visto che le DLL si possono caricare/scaricare a piacimento, i
processi non possono avere uno stato, specialmente se condividono codice, così
\i{}
da aumentare al massimo la condivisibilità. Il kernel Microsoft è un microa
kernel, proprio sfruttando le DLL (si caricano sempre e solo quelle necessarie).
Si utilizza un sistema di conteggio dei riferimenti per decidere se deallocare una
DLL.
Le DLL sono quindi ascrivibili in 3 diversi gruppi: kernel, utente (windowing) e GUI. Il sistema è però facilmente
estendibile, potendo introdurre nuove
DLL (es, WinSocket).
Nei sistemi operativi Microsoft tutti i tipi di oggetti (sia utente che kernel)
possiedono degli appositi handler per riferirli ed identicarli, in maniera tale da
poterli utilizzare anche con le apposite API di Windows. A differenza di Unix,
l'interfaccia per richiamare gli oggetti non è molto compatta, risultando essere
molto complessa.
Windows fa un uso pesante del modello ad eventi (il sistema graco fa un
buon match con ciò). Si tratta di un modello fortemente push (all'epoca era
anche l'unico modello esistente per lavorare quando non si avevano ancora i
processi). Gli eventi vengono usati ancora anche a livello applicativo: a livelli
inferiori risultano essere molto più essibili, sfruttando il modello a call-back:
l'applicazione fornisce un handler/funzione da richiamare nel caso si presenti
l'evento alla DLL opportuna34 . L'evento è lanciato da una qualunque intere
azione dell'utente o presso una periferica esterna, e viene lanciato a tutti quelli
che lo necessitano (ogni programma ha una sua coda per gli eventi).
Un'altra tecnologia importante introdotta dalla Microsoft è il cosidetto Obe
ject Linking and Embedding, per cui si vuole gestire le applicazioni in maniera
34 Quindi
ogni finestra registra per ogni evento un possibile handler!
72
integrata. L'idea è quella di suddividere correttamente i compiti fra il docue
mento ed il gestore del documento. Ogni tipologia di documento, infatti, ha un
proprio gestore che viene richiamato all'atto del caricamento del documento.
Tuttavia, un documento può essere fatto di diverse parti, ovvero di altri
documenti: si deve quindi anche caricare il gestore degli altri documenti di cui
è composto. In particolare si può avere che:
$\bullet$ Il documento contiene un riferimento all'altra tipologia di documento,
ovvero si fa linking
$\bullet$ Il documento contiene proprio una copia dell'altro documento, ovvero si
fa embedding. Quest'ultimo caso non è realizzabile direttamente con il
modello ad oggetti!
Mediante questa tecnologia si può quindi attivare un'applicazione all'interno
dell'altra: vi possono essere problemi di duplicazione dell'oggetto o la presenza
di riferimenti multipli e di molteplici gestori della stessa tipologia di documento
(quanti ne attiviamo?). Tuttavia, nel concentrato non presenta grossi problemi
e funziona bene.
L'idea della Microsoft era però quella di estendere il concetto di OLE anche
al distribuito, da qui la tecnologia COM. Nel concentrato era tutto più semplice
perché si sfruttava il registry di Windows per identicare in maniera univoca un
oggetto35 , e per poter registrare quindi i vari gestori.
\subsection{Architettura COM}
COM è stato il sistema standard per la Microsoft per poter descrivere l'intere
azione fra processi diversi, utilizzando come base i componenti e una comunicazione C/S sincrona. Si può notare la
differenza con i precedenti modelli: nei
sistemi precedenti la comunicazione si basava soprattutto sul modello ad eventi,
che è molto lontano dalla comunicazione sincrona!
Lo standard definisce delle specifiche per cui i componenti siano in grado
di comunicare fra di loro, in maniera indipendente dal linguaggio utilizzato e
dalle applicazioni che li usano. L'idea è quella di introdurre una uniformità nel
comportamento: dll, componenti su macchine diverse, . . . Si ottiene così omo\i{}
geneità a livello di comportamento fra oggetto locale e remoto, introducendo
così trasparenza. In particolare, si utilizzano metodi standard per la comuni\i{}
cazione remota, come RPC di DCE.
Alla base di COM vi è l'uso delle interfacce: un componente, per poter
essere usato, deve fornire un'interfaccia comune e conosciuta da tutti gli attori. L'interfaccia infatti definisce la
visione logica. Tutte le interfacce COM
derivano da una stessa interfaccia detta IUnknown. In COM ogni oggetto e ogni
interfaccia possono essere identicati in maniera univoca da dei GUID, che però
distinguono anche fra componenti ed interfacce.
35 Sfruttando il GUID, derivato dall'UUID descritto dal DCE. In Microsoft, i diversi GUID
son classicati in base al tipo di oggetti che riferiscono (interfacce, classi, . . . )
73
L'interfaccia funziona semplicemente come un puntatore alle funzioni che descrive, contenute in una virtual method
table: ogni entry di questa tabella punta
al codice che i metodi devono implementare. In COM non esiste l'ereditarietà
a livello di classe ma solo (anche multipla) a livello di interfacce. Si ha quindi
un grafo che rappresenta l'ereditarietà, che spesso è difficile da navigare! I nomi
univoci sono infatti tutti registrati nel registry, a seconda della loro applicazione.
L'interfaccia IUnknown presenta solo 3 metodi (presenti sempre nei primi
3 slot di ogni VMT di ogni componente): uno per aggiungere un riferimento
all'oggetto, uno per farne il release, e inne un QueryInterface. Quest'ultimo è
fondamentale per poter interagire con l'oggetto stesso: prende infatti il GUID e
verica se esiste un oggetto che implementi quell'interfaccia. In caso aermativo,
si preoccupa di restituire un puntatore all'interfaccia corrispondente. Questo è un metodo pervasivo, è l'unico modo con
cui accedere ai diversi metodi. Gli
altri infatti sono utilizzati per gestire le risorse in maniera dinamica.
La deallocazione degli oggetti può essere decisa: o si utilizza un garbage colo
lector, oppure si utilizza il sistema del reference counting, oppure si può anche
non gestire! Si può quindi osservare una differenza rispetto a CORBA: qui è
necessario vedere come avviene l'interazione di basso livello! Si deve per forza
passare per il metodo QueryInterface.
In COM vi è quindi solo l'ereditarietà in base alle interfacce: questo perché
l'ereditarietà all'epoca era vista come un sistema troppo accoppiante, e quindi da
limitare. Infatti, ereditando solo le interfacce non si eredita l'implementazione
ed eventuali comportamenti non desiderati. Fondamentalmente, l'ereditarietà
rompeva il principio dell'incapsulamento (la classe derivata infatti può accedere
alla classe base)!
COM quindi introduce due metodi alternativi per poter estendere una classe:
aggregazione o delegazione. In ogni modo, un oggetto non implementa tutti i
metodi che la sua interfaccia espone, ma {`}ridirige' la richiesta di controllo ad
oggetti che implementano quei metodi (in un qualche modo li incapsula).
La delegazione indica che fra gli oggetti vi deve essere una condivisione
del comportamento, ma non dell'implementazione (non vi è l'obbligo di usare
la stessa implementazione). La delegazione fondamentalmente corrisponde ad
esplicitare il fatto che vi è un altro oggetto che si preoccupa di rispondere
alla richiesta in maniera corretta. Questo è un oggetto interno, che deve esere
sempre attivo, e puntato in maniera opportuna dalla VMT dell'oggetto esterno!
Si tratta sicuramente di uno dei fattori di maggiore complessità dell'architettura
COM.
L'aggregazione invece gestisce il problema in maniera implicita, realizzando
quindi una serie di oggetti aggregati. Non è come prima, per cui l'interfaccia
esterna funzionava come un passacarte, ma si realizza come una copia dell'interfaccia dell'oggetto aggregato nella VMT
dell'oggetto esterno! Fondamentalmente, quindi, l'oggetto aggregato fornisce i propri metodi all'esterno, direttamente
nell'interfaccia dell'oggetto esterno! Anche questa maniera è complessa,
più dell'ereditarietà. L'aggregazione è consigliata per una gestione a parti (a
74
differenza della delegazione, ogni oggetto si preoccupa di fare da gestore delle
richieste ai metodi che implementa).
\subsection{Interazione C/S in COM}
L'idea base è che vi sia uniformità di comportamento fra ogni tipologia d'oggete
to: non interessa quindi come è fatto, ma la possibilità di poterci interagire in
maniera C/S trasparente. Il problema è che il costo è molto diverso in realtà a
seconda del tipo di oggetto con cui si prova a colloquiare (accedere ad una DLL
in maniera C/S è molto più ottimizzato che accedere ad un oggetto remoto).
Quindi, a differenza di CORBA36 , l'implementazione sottostante è diversa a
seconda di quale tipo di oggetto si acceda
Un oggetto in COM, perché si possa riconoscere come server, deve fornire
la capacità di istanziare degli oggetti : in generale questo viene realizzato utiliza
zando delle Factory, che non son classi ma gestori d'entità. Ogni Factory ha
una sua politica per l'attivazione (potrebbe per esempio realizzare un singleton,
fornendo il riferimento sempre alla sola copia attiva), ma i diversi meccanismi
devono restare nascosti.
Un server quindi implementa o l'interfaccia IClassFactory (oppure la sua
seconda versione, che permette di lavorare anche sfruttando un'apposita licenza dell'oggetto stesso). Queste interfacce
infatti definiscono un metodo per
creare un'istanza dell'oggetto, oppure di ottenere un riferimento all'oggetto già
esistente (CreateInstance). Questo metodo (e le sue varianti) devono quindi
esplorare il registry alla ricerca di un oggetto che implementa l'interfaccia ricercata; si può anche specificare un
contesto, ovvero spiegare dove si potrebbe
ricercare l'oggetto.
Esistono 3 tipi possibili di server in COM:
$\bullet$ In-process: lavora nello stesso spazio di lavoro del client, mediante un'apposita DLL.
$\bullet$ Locale: lavora in un processo diverso, ma sulla stessa macchina.
$\bullet$ Remoto altrimenti.
Il problema di COM è che, volendo garantire un'alta trasparenza, richiedeche
vengano definite tantissime interfacce. Un'altra interfaccia è per esempio COe
MOBJ, che si preoccupa di cercare nel registry l'esistenza delle DLL necessarie,
ne fornisce il riferimento o le carica in memoria. Questo comporta che l'utente debba conoscere i meccanismi presenti in
COM ! Fortunatamente, gli stessi
meccanismi funzionano anche lavorando out-of-process.
L'unica cosa che si aggiunge lavorando out-of-process è la presenza di proxy
e stub per facilitare la comunicazione e realizzare la trasparenza alla comunicazione. Fra proxy e stub si stabilisce
quindi in maniera automatica un canale,
che permette ai due processi di comunicare fra di loro: è comunque un'astrazione
36 Uso
perenne dell'ORB
75
(in locale si lavorerebbe utilizzando la memoria comune!). Per definire questi
componenti, si è sviluppato un apposito IDL, MIDL. Questo deriva e rispetta
le specifiche di DCE, ma risulta comunque essere incompatibile con altri IDL.
Come funziona quindi la comunicazione in COM? Si lavora per sequenza: si
tenta inizialmente di accedere alla risorsa in process, poi in locale, ed inne si
tenta in remoto sfruttando come meccanismo RPC. In certi casi, essendo poco
costosa, si utilizza RPC anche in locale, di tipo diverso e più leggera.
Per la gestione remota, vi è anche la presenza di un ulteriore attore al livello
più basso detto Service Control Manager, che si comporta proprio come un mw!
Infatti, un proxy che volesse comunicare, si riferisce al SCMper ottenere un
riferimento all'oggetto remoto, e quindi può realizzare una RPC remota.Questa
architettura comporta quindi che SCM debba essere presente su ogni nodo, e
che proxy e stub debbano essere stabiliti prima, in maniera da poter conoscere
la posizione del SCM! Si potrebbe anche avere, tuttavia, che non vengano mai
usati se la risorsa è disponibile in-process!
\subsection{DCOM}
DCOM nasce negli anni 90 come esigenza si introdurre anche in COM (come
in CORBA) l'interazione dinamica. In particolare, ciò che si introduce è il
concetto di automation, ovvero la possibilità di fornire ad un client di poter
gestire oggetti per cui non aveva previsto l'interazione. Si tratta anche di un
modello realizzato per cercare di andare oltre al classico modello a code di eventi.
Inizialmente, automation nasce come possibilità per linguaggi dinamici (tipo
quelli di scripting per il Web) di poter accedere all'architettura COM. Questo
genere di linguaggi infatti spesso sono loosely-typed e studiati in maniera tale
da avere poco controllo. Prima di automation risultava difficile realizzare proxy
anche per questi linguaggi.
Quello che fa automation è semplicemente la realizzazione in automatico di
una DLL che funzioni come intermediario fra i due mondi, fra COM e qualsiasi
altro lingaggio, se tale interazione non era già stata prevista in maniera statica.
Automation permette quindi di esplorare, in maniera automatica, un oggetto
la cui interfaccia è sconosciuta a tempo di esecuzione. Automation incorpora
concetti già visti in OLE, come la in-place activation, per cui un gestore viene
attivato solo se è necessario, cioè se è richiesto da un sottocomponente. DCOM
e e
introduce anche altre tecnologie, come lo structured storage (tutto è all'interno
di un singolo le) e un sistema di scambio dei dati unicato. DCOM introduce
soprattutto diversi strumenti proprietari Microsoft per il web, spesso incompatibili con altri strumenti.
Per introdurre la dinamicità, un oggetto deve implementare anche l'interfaca
cia IDispatch, che è riflessiva, la base stessa di automation. Infatti, a differenza
di CORBA, in COM-DCOM non vi è un IR per registrare le interfacce: si tratta
quindi di interrogare ed invocare in maniera dinamica i componenti! Si tratta
di un sistema del tutto analogo alla Request di CORBA.
76
Quindi, alcuni metodi riportano informazioni sul componente (GetTypeInfo,
GetTypeInfoCount), altri preparano il client a richiedere l'invocazione (GetIDsOfNames) e inne vi è un metodo per
invocare in maniera dinamica l'oggetto
stesso. IDispatch, essendo un'interfaccia COM, eredita direttamente da IUnknown. Possiamo così sostituire questa ed
altre interfacce, permettendo un'ese\i{}
cuzione sempre dinamica (ma costosa).
Si può quindi iniziare a parlare di componenti veri, che rappresentano un'esteno
sione degli oggetti: sono più pervasivi, garantendo di non avere limiti alla riusabilità e possibilità di essere usati
in un qualunque contesto, senza dover dipena
dere come gli oggetti da un determinato linguaggio.
A differenza però di CORBA, essendo la VMT ssa, non è possibile realizzare
un server dinamico!.
\subsection{Interazione in DCOM}
COM era stato pensato per fornire una visione di alto livello uniforme (sempre
C/S), per cercare anche di rendere un'ottimizzazione molto forte a seconda
della comunicazione che si andava a realizzare. A basso livello invece si ha un
comportamento molto variegato, con un'alta idea di delegazione.
A sua volta, DCOM prevede di definire dei riferimenti diretti fra C e S,
facendo in modo quindi che si conoscano comunque, nonostante l'uso di un'interfaccia. Questi riferimenti quindi non son
persistenti.37 . DCOM si preoccupa
di aumentare la dinamicità, cercando anche di andare oltre al C/S, ma fornendo
strumenti per trovare in maniera automatica la soluzione, fornendo supporto
alla portabilità, cercando di dare una visione semplificata. Purtroppo, per reala
izzare ciò si è dovuto ripensare l'architettura, mantenendo però un sistema per
o e
integrare il sistema legacy precedente.
Quello che infatti succede è che in automation le DLL diventano oggetti ade
hoc. Questi oggetti, riassunti ed estensioni, sono i cosidetti ActiveX. Includono
anche idee come OLE, ma restano molto complessi da realizzare, per cui è
necessario sfruttare degli appositi wizard. Sono quindi fondamentalmente dei
piccoli server OLE, in grado di attivarsi con il metodo della in-place activation,
ma con un comportamento diverso: infatti, lavorano in maniera inside out,
ovvero son sempre pronti, attivi, no a che l'oggetto contenuto è presente. Si
tratta quindi di un sistema ad aggregazione sempre attiva (il componente interno
è sempre attivo).
\subsection{Standardizzazione dei componenti}
Un merito di COM e DCOM è stato quello di fornire un primo tentativo di stane
dardizzare il concetto di componente. Prima si dovevano definire sempre diverse
37 Vi è una losoa diversa rispetto a CORBA: qui è necesario avere anche una visione di
basso livello, nonostante l'uso delle interfacce come in CORBA
77
interfacce, sempre utilizzabili dagli utenti, ma senza una razionalizzazione alla
loro realizzazione.
DCOM definisce per primo che un sistema a componenti deve definire un'architettura a PEM (Property, Event, Method ), per
poter definire stato, output
e input dei componenti. Questo sistema semplica la realizzazione, permettendo quindi di definire un container che sia
semplicemente un gestore degli oggetti
contenuti, in grado di attivarli attravero un'interfaccia uniforme e al sistema dell'introspezione. Il container deve
solo fornire un'interfaccia IDispatch! Quindi,
il container implementa la dispinterface definita dall'oggetto contenuto (questo
deve solo definire quali eventi sono interessanti per lui); uno è la sorgente dele
l'interfaccoa, l'altro il pozzo che la realizza. A sua volta, il container sfrutta
degli oggetti intermedi per poter comunicare con l'oggetto contenuto.
DCOM stabilisce anche la possibilità di realizzare una persistenza del coma
ponente, implementando un'apposita interfaccia. (I riferimenti non son ssi in
DCOM).
\subsection{Gestori delle sottoparti: i monikers}
DCOM introduce i monikers come un sistema ad hoc, piuttosto complesso, per
poter gestire in maniera persistente la gestione degli oggetti. Questo sistema
aumenta la flessibilità, memorizzando su disco quindi i riferimenti agli oggetti,
categorizzandoli in base a cosa sono, e rendendoli indipendenti dalle applicazioni!
Si ha quindi che ogni moniker viene associato ad un programma, permettendo
anche di lavorare mediante nomi indiretti. Permette anche di moltiplicare i
meccanismi, aggiungendo diversi monikers.
\subsection{Modello a thread di DCOM}
Il modello deriva da quello di Win32. Tuttavia, vi possono essere dei problemi se certi componenti non risultano essere
thread-safe (magari perché sono
componenti legacy). Per superare questo problema, in DCOM si è introdote
to il concetto di Apartment, che consiste ad un ambiente d'esecuzione isolato,
cioè con una politica di threading ssata. Vi sono due diversi modelli, o quello
single-threaded (più diffuso, un unico componente lavora nell'apartment) opu
pure multi-threaded. Ogni componente può quindi avere più STA, ma sempre
un solo MTA.
\subsection{Altri strumenti DCOM}
Microsoft ha fornito mediante DCOM tantissimi strumenti, in grado di risolvere
i problemi sempre più nell'ottica di un moderno mw.
In particolare, si possono citare MSMQ per realizzare un sistema a scambio di
messaggi, e COM+ per permettere la denizione di componenti solo attraverso
la parte logica. Presenta anche un servizio X.500 di directory standard, detto
Active Directory.
78
\subsection{.NET}
Questa è la nuova architettura Microsoft, sviluppata ripensando totalmente il tutto! Non solo si è progettato pensando
ad un S.O. diverso, ma anche rivoluzione 
nando la stessa macchina virtuale (realizzando il CLR). Si tratta di un sistema
derivato e migliorato (sotto certi aspetti) da Java: Java infatti ormai presenta
dei sistemi legacy con cui deve far conto, per cui non può introdurre direttamente
nuovi aspetti senza rompere con il passato (non si può, per esempio, introdurre
un controllo dinamico delle risorse occupate da un thread, per denirlo anche a
run-time!). .NET realizza anche lui un compilatore JIT, definisce una serie di
classi base, rintroduce l'ereditarietà dell'implementazione, delle classi, e svilupa
pa un sistema a livelli molto simile a quello di un mw moderno, con diversi
servizi attivabili a basso ed ad alto livello.
In particolare, .NET introduce una forte idea di località sfruttando l'idea
dell'application domain: i processi non possono comunicare in maniera diretta
fra di loro, ma devono sfruttare dei protocolli DCOM compatibili (è un sistema
simile all'apartment), oppure WS o altre nuove tecnologie .NET. WS per esempio è la tecnologia più costosa, ma anche più
vicina all'applicazione da un punto
di vista logico.
\subsection{.NET remoting}
Si basa comunque sull'idea di realizzare dei proxy per facilitare la comunicazione.
Tuttavia, vi sono due diversi tipi di proxy: quello trasparente (utile per fare
richieste direttamente al supporto .NET e generato automaticamente) e quello
reale (realizzato per facilitare la comunicazione C/S, spesso presente). Si ha
quindi che il secondo proxy è anche modicabile, per poterlo adattare alle proe
prie esigenze. In particolare, con questo sistema, si riesce a registrare un oggetto remoto all'interno del proprio
application domain, per poterlo utilizzare in
maniera trasparente!
I proxy sono utilizzati per superare l'eterogeneità fra i sistemi, e per generare
un canale come veicolo di trasporto dell'informazione. Questi canali possono essere di diverso tipo (binari, testuali,
http compatibili, . . . i 2 standard sfruttano
o TCP o HTTP), e si possono combinare per realizzare una catena di responsabilità (formattazione del messaggio,
encoding, . . . )! Sono questi oggetti che
permettono la comunicazione fra application domain diversi. Ogni application
domain quindi registra i canali che si possono usare, permettendo alle volte anche una comunicazione bidirezionale.
Di default, le operazioni remote sono sempre sincrone, ma sfruttando opportunatamente i delegati si possono ottenere
anche interazioni sincrone non
bloccanti, sfruttando un meccanismo di call-back. I metodi possono fornire un
passaggio per valore (maggiormente disaccoppiato, ma in presenza di dati grandi, aumenta notevolmente l'overhead) o per
riferimento (mediante la creazione
di un proxy per memorizzare quindi un riferimento remoto), in maniera quindi
79
del tutto simile a CORBA!
L'attivazione a chi tocca? Non son presenti dei POA come in CORBA,
quindi si potrebbe prima di tutto pensare ad un'attivazione da parte del cliente
(esistono meccanismi per fare il lease, e quindi mantenere il collegamento con il
server oltre al tempo specificato38 ). In questa maniera si può anche mantenere
uno stato.
Gli oggetti remoti possono anche essere attivati dai server, in due diversi
modi:
$\bullet$ Come singleton (si attiva ora per sempre, ma non si può garantire uno
stato permanente: il controllo dell'oggetto resta in mano al supporto che
potrebbe deallocarlo).
$\bullet$ oppure come oggetti di tipo single call, ovvero con durata limitata al tempo
dell'invocazione, e quindi di costo limitato.
Si ha quindi una politica d'attivazione molto più semplice rispetto a quella di
CORBA!