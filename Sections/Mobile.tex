\section{Gestione delle risorse e sistemi mobili}
Le risorse in un sistema distribuito possono essere o concentrate o distribuite.
L'idea è comunque quella di cercare di orire un servizio d trasparenza accedene
do alla risorsa.
Si deve quindi avere un sistema per descrivere le risorse, magari sfruttando
l'XML se si sta lavorando ad alto livello. Se inoltre si potesse definire la qualità
a
sulla risorsa, si potrebbe dimostrare che si è capito il costo fondamentale nello
e
sviluppo.
Si possono avere due diversi piani per il progetto delle risorse:
$\bullet$ Statico: si progettano le risorse no al deployment, realizzando anche delle
politiche complesse per poter descrivere dove inserirle, come replicarle,. . .
$\bullet$ Dinamico: si realizzano politiche per la gestione e il deployment durante
l'esecuzione dell'applicazione stessa.
Tipicamente, si ha già pensato a come realizzare il deployment delle risorse, alla
a
loro allocazione, essendo l'approccio più semplice da gestire: questo però è un
u
oe
approccio poco essibile e dinamico.
Un esempio di sistema distribuito coordinato per le risorse potrebbe essere
un File System distribuito: richiederebbe l'introduzione di agenti coordinati per
gestirle (questi dovrebbero introdurre delle fasi di negoziazione, prima di poter
fornire l'accesso alla risorsa). Garantisce trasparenza all'allocazione delle risorse,
mantenendo un comportamento uguale. Un altro modello potrebbe essere quello
basato su una service request, che rappresenta un sistema C/S.
\subsection{Mobilità delle risorse}
a
Nel progetto di un sistema, si deve considerare cosa e come si può muovere. Per
o
esempio, se si pensa allo stack di un programma Java, questo è legato alla stack
e
della JVM: non si può spostare facilmente su un altro nodo! Altre caratteristiche
o
dicili da muovere sono i riferimenti a risorse strettamente locali (tipo i le40 ).
Se è presente la possibilità di muovere le risorse, aggiungendo un servizio di
e
a
trasparenza si ha che si lavora senza fornire la possibilità di vedere le risorse
a
locali!
Questi concetti sono utili anche per decidere cosa posizionare su ogni nodo:
se per esempio le risorse dovessero essere allocate tutte sullo stesso nodo, si può
o
40 Si
potrebbero spostare, se si lavorasse su un le system distribuito
84
realizzare un sistema a memoria comune, con operazioni però solo sequenziali.
o
Potendole distribuire su più nodi, si riescono a fornire operazioni anche in paru
allelo ma molto più costose. Se le risorse son semplici poi, basta semplicemente
u
copiarle, mentre certe risorse con vincoli che le rendono complesse, richiedono
un apposito sistema di trasparenza.
Esistono quindi diversi requisiti per fare un'esecuzione remota: si deve cercare di limitare l'overhead, e di
interferire il meno possibile con l'esecuzione
locale (gradita la non-interferenza), ed è necessario fornire anche delle infore
mazioni di stato sui singoli processori. Questo è particolarmente importante,
e
vista l'ampia eterogeneità presente: non vi è per esempio un sistema di nomi
a
e
comune, ed è quindi necessario stipulare delle apposite convenzioni.
e
\subsection{Muovere i processi}
Esistono diversi S.O. che prevedono la possibilità di spostare i processi su altri
a
processori, per cercare di fornire un bilanciamento di carico migliore. Si realizza
così uno scheduling distribuito, in grado di fornire sia una politica locale che
\i{}
globale con cui trattare i processi. Un sistema di scheduling globale deve essere
in grado di fornire dei meccanismi con cui i processi possano comunicare in
remoto e quindi la possibilità di realizzare delle politiche per la gestione delle
a
risorse.
In particolare, si possono avere due diversi approcci:
$\bullet$ Load Sharing: si denisce la condivisione del carico come carico globale
del sistema, ovvero è uno scheduling totale del sistema. L'idea è quella di
e
e
evitare processi idle, mantenendoli sempre al massimo del lavoro possibile.
Si tratta di una valutazione statica, i processi non si possono muovere su
altri processori
`
$\bullet$ Load Balancing: E una valutazione dinamica, ovvero a tempo d'esecuzione
le risorse si possono muovere. Si vuole quindi mantenere un carico equilibrato su ogni processore, per ottenere
un'efficienza elevata.
Il problema del muovere un processo è che è un'operazione particolarmente cose
e
tosa. Utilizzando una politica di load balancing, conviene comunque fare delle
valutazioni dinamiche (quindi non di tutto l'insieme dei processori; spesso sono
valutazioni euristiche, per poter limitare il costo a volte inaccettabile) per approssimare il comportamento, per
limitare il costo. Una possibile strategia, per
esempio, è quella di ridurre il numero dei processori su cui un processo può ese
o
sere trasferito. Il load balancing deve essere valutato attentamente, perché può
e o
divenire molto intrusivo. Le valutazioni dinamiche, che realizzano politiche di
minima intrusione locali e semplici son sempre da preferire.
Si sono visti diversi modelli per l'esecuzione di processi per ilload sharing:
$\bullet$ Modello a ring logico: un esempio è V-kernel. Si denisce quindi una
e
struttura ssa, statica, che spiega come i processi possano muoversi. Vi
85
` un token che stabilisce quale processore è il gestore in quel momento, il
e
e
quale richiede informazioni sul sistema mediante un broadcast. Ottenute
le risposte, si genera il bilanciamento distribuendo il carico (che non si
`
può quindi spostare). E una struttura statica e proattiva, però semplice
o
o
da gestire anche a fronte di guasti.
$\bullet$ Modello a foresta: un esempio è Micros. Vi è una gerarchi di processori, in
e
e
genere rappresentata ad albero, quindi più dinamica. Possibilità di rappreu
a
sentare anche foreste di processori. Al livello più basso vi sono i processori
u
che lavorano direttamente sulle risorse (worker), e i loro padri sono i manager: la profondit` dell'albero dipende
quindi dal numero delle risorse che
a
si vengono a definire (in generale si cerca di avere un albero binario). L'obiettivo di Micros è quello di gestire un
elevato numero di risorse ed utenti,
e
non basandosi sulla topologia reale della rete. Il sistema è dinamico proe
prio perché permette anche di fare un'allocazione dinamica delle risorse,
e
richiedendole al livello superiore. L'architettura è fault-tolerant.
e
$\bullet$ Modello derivato dai worm d'Internet: si tratta di un buon livello per
la ridistribuzione del carico. Rappresenta un sistema molto dinamico che
richiede pochissime informazioni sull'architettura del sistema. L'idea `
e
quella di ricopiare l'applicazione che si ricerca, valutando i nodi vicini e
scegliendo quelli liberi.
Nel caso di load balancing si parla di migrazione: un processo viene trasferito su
`
un altro nodo. E un'operazione costosa, ma che presenta degli indubbi vantaggi,
perché permette di smistare il carico nel sistema. In questo modo si riescono a
e
far lavorare tutti i processori, aumentando l'efficienza.
`
E ovvio però che per capire se le politiche addottate sono veramente un aiuo
to, serve un apposito sistema di monitoring: rappresentano dei costi aggiuntivi,
ma necessari per poter valutare istante per istante la situazione del sistema, e
quindi per poter bilanciarlo correttamente! Il monitoring deve tenr conto dei
processori, delle risorse e del tempo di comunicazione, e deve essere studiato per
realizzare una politica di minima intrusione, assumendo un principio di continuità dell'applicazione. Si introduce così
una logica di trasparenza, rendendo
a
\i{}
certi dati noti solo al supporto. Si punta quindi all'efficienza del sistema, alla
mobilità e alla realizzazione di un sistema fault-tolerant (crolla un processore,
a
ma il processo migra su un'altra macchina!).
La migrazione deve tener conto di diversi aspetti:
1. Si deve comunque lasciar la precedenza alle computazioni locali.
2. Si deve evitare il fenomeno del trashing, ovvero che un processo si muova
continuamente da un processore all'altro, senza mai essere eseguito.
3. Si deve evitare il fenomeno delle dipendenze residue: un nodo non dovrebbe
mai ridirezionare le richieste verso il nuovo nodo su cui gira il processo.
Vi è infatti a possibilità di generare dei cicli assolutamente pericolosi.
e
a
86
4. Il sistema deve essere concorrente, in grado di realizzare migrazioni multiple.
Si deve quindi progettare il processo in maniera da poter trasportare quello che
serve: in generale lo stato corrente più eventuali modiche che sono state apporu
tate (magari si può trasportare anche solo un sottoinsieme delle informazioni).
o
Il grosso problema della migrazione è quello del riuscire, nella maniera più
e
u
eciente e meno costosa, a fornire ai clienti un sistema per accedere al processo
anche se è stato spostato.
e
Una prima possibilità è quella di ridirigere i messaggi, sfruttando per esemae
pio un'apposita struttura detta forwarder, in grado di farlo in maniera traspar`
ente. E una strategia pessimistica/pro-attiva. In questo modo il cliente non
viene avvisato! Una prima modica consiste quindi nel riqualicare i messaggi,
avvisando anche il cliente oltre a ridirigerli per un certo tempo.
Tuttavia, la strategia più brusca (ottimista e reattiva) è quella di far proprio
u
e
fallire il client: in questo modo il client si accorge che vi è un qualche problema
e
(non è trasparente!) e si deve preoccupare di ricercare dove è adesso il processo
e
e
che desidera, per ottenerne un riferimento.
Demos/MP (anni 70) è stato uno dei primi S.O. a pensare ad un sistema
e
`
di load balancing, in cui è il supporto che fa migrare i processi. E un sistema
e
basato sullo scambio di messaggi, e i processi si conoscono in base ad appositi
link : queste strutture non sono come le porte, non son legati ai nodi su cui
girano, ed era un sistema univoco per puntare al processo. Si tratta quindi di
sistemi gestiti dall'infrastruttura.
In particolare, Demos/MP si preoccupa di spostare solo processi pesanti, cioè
e
che hanno un tempo d'esecuzione molto lungo (per esempio, processi ciclici).
Non si fanno quindi migrare i processi con un tempo di vita limitato.
Demos/MP è stato anche il primo ideatore di un forwarder, che può lavorare
e
o
correttamente grazie al sistema a link! Il forwarder tuttavia è una struttura
e
temporanea, nch` il link al processo non viene riqualicato, ovvero anche i
e
client riescono ad accederci direttamente senza dover passare dal nodo iniziale
(si può comunque anche fare un sistema a scarto di messaggi, per cui la reo
sponsabilit` del trovare il nuovo processore tocchi al cliente). Si deve quindi
a
fermare l'esecuzione del processo sul nodo originale, trasferire lo stato sul nuovo
processore, attivarlo e attivare il forwarder temporaneo. Si ha quindi alla ne
una trasparenza all'allocazione.
Un altro S.O. che prevede la migrazione dei processi è il V-kernel (sempre
e
a scambio di messaggi), che punta all'efficienza e ad essere fault-tolerant, oltre
alla trasparenza all'allocazione. Questo S.O. lavora in maniera preventiva, per
cui copia le dierenze fra i processori su cui gira (in tempi diversi) per evitare
le dipendenze residue dovute al forwarder! Si ha quindi sempre una riqualicazione dei link.
87
La migrazione è un modo molto buono per poter separare meccanismi e
e
politiche. Le politiche per la migrazione dipendono da diverse considerazioni:
1. Vale la pena migrare?
2. Chi e quando trasferisce il processo?
3. Su quale processore si allocher` il nuovo processo?
a
In generale possiamo osservare che vi sono dei meccanismi/caratteristiche comuni, per cui vi sono solo certe categorie
di processi che si possono spostare, ed
` necessario un gestore per la migrazione per ogni nodo. Si deve quindi essere in
e
grado di definire cosa trasferire, cosa si può migrare, ovvero la costituzione steso
sa della risorsa! Si ha che quindi si blocca il processo, se ne crea una copia sul
nuovo processore, e quindi si avvia un sistema per riqualicare i link, elimando
quelli obsoleti. Nel mentre, si riutano le comunicazioni con il processo.
Quanto costa? Un S.O., Charlotte, stimava il costo in base al numero dei
link presenti, della dimensione del processo da trasferire e da una parte costante
di comunicazione: al crescere delle dimensioni del processo, il tempo totale per
il trasferimento aumenta notevolmente.
Le politiche di migrazione sono quindi stabilite mediante un sistema di valutazione del carico, e la decisione di chi
trasferire e quando, spesso accomunata
a dove si deve trasferire, ovvero un concetto di locazione. In particolare, si deve
trovare un sistema per valutare qual è l'impatto sullo scheduling locale, potendo
e
integrare quindi politiche di più alta gestione. Le politiche si possono quindi
u
classicare come:
$\bullet$ Statiche: sono molto facili da valutare, e risulta facile definire come trasferire
(tipo, si decide che sono i processi nuovi a muoversi, fornendo però un rio
schio di trashing). La locazione è statica, e anche se non risulta essere
e
essibile, presenta un costo molto basso.
$\bullet$ Semi-dinamiche: sono politiche più costose, perché dipendono dalla situu
e
azione attuale. Si realizza una scelta ciclica dei processi e dei destinatari.
Il carico è quindi dinamico.
e
$\bullet$ Dinamiche: Si intende di lavorare per un sistema di vicinato, ovvero si
cerca di spostare il carico su processi vicini. Si deve quindi valutare il
carico del vicinato, e trovare sistemi poco costosi per determinarlo.
Le politiche possono essere più o meno complesse, realizzando magari politiche
u
incondizionate (meccanismi random) o condizionate: queste ultime sonopiù cosu
tose, presentando un overhead dovuto ad una comuncaizione per fare delle negoziazioni ! Questo sistema è detto probing,
perché ovviamente si cerca nel
e
e
vicinato, per limitare il costo (non ha senso cercare nella globalit` !). Si possono
a
anche realizzare politiche di bidding, per ricercare un processore disponibile ad
eseguire il processo, e scegliendo la migliore oerta proposta.
88
Non esiste una politica migliore in assoluto: in generale, conviene avere
un'inziativa da parte di un sender (da chi possiede il processo e che lo vorrebbe
spostare) se si ha un sistema con carichi bassi, altrimenti da parte del receiver
(un processore che potrebbe fornire risorse per un altro processo). Un approccio
misto risulta quindi essere il migliore.
La migrazione è un sistema che deve essere studiato attentamente: può
e
o
infatti ridurre notevolmente i tempi dell'applicazione, sfruttando politiche semplici. Si ha quindi un'intrusione
limitata per ottenere i risultati desiderati. Si
deve quindi puntare all'efficienza, tendere all'ottimalit` e realizzare un sistema
a
stabile.
\subsection{Sistemi ad agenti mobili}
I sistemi ad agenti mobili si mappa molto bene con il problema della migrazione.
Tuttavia, l'approccio è diverso: non si navigano i nodi per ottimizzare il bilane
ciamento del carico, ma per ottenere informazioni dai diversi nodi. Si ha quindi
un movimento dovuto all'applicazione, e non da esigenze di ecienze per la
computazione e diretto dal supporto!
Il movimento è quindi una caratteristica base degli agenti mobili, per cui si
e
deve cercare di determinare dei sistemi ecienti. Nulla per esempio impedisce
ad un agente di ritornare su nodi già visitati.
a
Per gli agenti si parla quindi di mobilità del codice (modello che va oltre
a
al normale approccio C/S, per cui si passano solo dati. Sono sistemi utili per
esempio per aggiornare i diversi router). Prima degli agenti mobili, per mobilità
a
del codice si parla di:
$\bullet$ Remote EValuation: è un'operazione singola, one-hop, per cui si invia il
e
codice al server, che ne diventa parte integrante!
$\bullet$ Code On Demand : processo inverso, ovvero il client scarica codice proveniente dal server (un esempio classico
sono le applet Java!)
Questi sistemi però non prevedono ancora la possibilità che il codice, mentre si
o
a
muove, esegua! Questo si ha solo con l'introduzione degli agenti mobili. Sono infatti sistemi a multi-hop, in grado di
cambiare l'allocazione durante l'esecuzione
e mantenere comunque uno stato coeso. Si può pensare a varie ottimizzazioni,
o
magari trasportando solo le dierenze rispetto ad uno stato iniziale o una sintesi
dei risultati ottenuti.
\subsection{Classicazione della mobilità}
a
Per gli agenti mobili, si può classicare in diversi gradi la mobilità:
o
a
$\bullet$ Forte: In ogni punto del codice è possibile specicare la mobilità. Tali
e
a
sistemi sono in realtà dicilmente realizzabili, perché presentano grossi
a
e
problemi di supporto.
89
$\bullet$ Debole: Si vincola invece la posizione in cui si può mettere la {`}move': tale
o
sistema è meno essibile, ma maggiormente e facilmente gestibile.
e
La differenza quindi è nella continuità della mobilità (o avviene sempre, oppure
e
a
a
solo in certi momenti prestabiliti). Come risolvere la mobilità forte? L'idea `
a
e
quella di realizzare un sistema a codice intermedio, potendo così fornire su ogni
\i{}
nodo degli interpreti del codice intermedio (per poter superare così i diversi
\i{}
problemi dovuti all'eterogeneità).
\subsection{Servono gli agenti mobili?}
In generale, un progetto giustica la scelta della tecnologia. Tuttavia, se i vincoli
risultano essere sbagliati, si usa a sproposito la tecnologia.
Attualmente, per gli agenti mobili si è ancora alla ricerca di una killer applie
cation; in certi casi conviene ancora usare sistemi come REV o COD. La mobilità
a
attualmente è vista come il sistema migliore per poter accedere a risorse vine
colate, cercando di ottimizzare il sistema utilizzando delle operazioni locali. Vi
sono diversi tipi di mobilità:
a
$\bullet$ Utente nomade: non ha senso utilizzare gli agenti, ma conviene invece
gestire un sistema a repository centrale. L'utente vuole eseguire le proprie
applicazioni indipendentemente da quale macchina stia utilizzando.
$\bullet$ Terminali mobili : potrebbe aver senso utilizzare gli agenti. Si tratta infatti
dell'idea di realizzare terminali in grado di lavorare comunque ed ovunque
si trovino.
$\bullet$ Codice mobile: denitivamente utile utilizzare gli agenti.
\subsection{Gli agenti mobili}
Sono delle entità che si devono muovere per eseguire i propri compiti, operando
a
per conto di un principal. Si tratta quindi di sistemi in cui il programmatore stabilisce la mobilità, e in cui si
realizza una programmazione location-awareness,
a
cioè dipendente dal posto in cui si esegue. Possono essere utili quindi per poter
e
fornire un controllo locale alle risorse, ltrando quindi le richieste da parte di un
gestore centrale! Gli agenti mobili devono essere progettati secondo un'ottica di
leggerezza: essere semplici, single-threaded (Java sembra un'ottima tecnologia
per realizzarli). Sono in corso di sviluppo mw, orientati alla weak mobility:
studio quindi di strutture dati apposite per la mobilità, permettendo soltanto
a
delle richieste esplicite alla migrazione.
Vi possono essere diverse implementazioni: con connessione o connectionless,
comunicazioni sincrone o meno, possibilità di realizzare un sistema sincronizzato
a
di agenti o meno, e così via. Tuttavia, tutti i sistemi ad agenti introducono un
\i{}
enorme problema per la sicurezza (codice che naviga ed agisce in una rete. . . ). Si
potrebbe infatti avere del codice maligno che gira. Tuttavia, i controlli potrebbero anche impedire ad un nodo di
eseguire codice non maligno, scartando
90
via l'agente! Si deve quindi vericare mediante l'uso di diversi agenti se per
vericare il comportamento del nodo.
Un altro problema è sul fatto che il codice può essere letto senza problemi
e
o
dal nodo su cui esegue: si devono quindi anche garantire certe proprietà di
a
incapsulamento, perché non sia un nodo a modicare il codice contenuto da un
e
agente!
A cosa può essere utile tutto ciò? A realizzare mobile computing (utenti in
o
o
grado di muoversi e di mantenere uno stato/sessione costante, anche cambiando
terminale, ottenendo così la massima accessibilit`) ottenendo per esempio la
\i{}
a
generazione di reti spontanee (utili per esempio per creare reti per giochi, solo su
disponibilità), possibilità di coordinare utenti mobili in ambienti civili e militari,
a
a
sistemi location awareness. . .