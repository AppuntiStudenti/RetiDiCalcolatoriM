\section{Modelli di computazione}
Per poter sviluppare un sistema distribuito, è utile definire diversi modelli per l'esecuzione, in modo da poter
classificarli (modelli diversi possono avere vantaggi/svantaggi a seconda del sistema che si deve andare a realizzare).
La maggior parte dei modelli purtroppo sono troppo astratti per poter essere realmente utili da un punto di vista
ingegneristico, cioè non riescono a comprendere in toto la \textit{complessità} reale.
\newline

Le prime distinzioni che si possono fare sono su come si possono prendere le decisioni:
\begin{itemize}
 \item modelli \textit{statici}: le decisioni vengono prese prima dell'esecuzione. Risulta quindi essere meno
flessibile, e non si può variare nel corso dell'esecuzione, fornendo quindi poca qualità di servizio.
\item modelli \textit{dinamici}: in questo modello le decisioni risultano essere maggiormente costose, proprio perché 
si prendono mentre il modello è in esecuzione.
\end{itemize}
Per fare un esempio, il collegamento fra un client e un Name Server come deve essere stabilito? In maniera statica o
dinamica? E fra il client e server? Si tratta di problematiche da risolvere in fase di progettazione, tenendo conto 
dei costi. Un middleware introduce questi costi, fornendo possibilità statiche/dinamiche (non esiste un sistema
totalmente dinamico, realmente totalmente aperto).
Una seconda distinzione riguarda sulla reazione in caso di errori o concorrenza:
\begin{itemize}
 \item modelli \textit{preventivi}: si fornisce un comportamento garantito, oppure si previene/evitano determinati
errori. E quindi un sistema rigido, che presenta dei costi fissi.
\item modelli \textit{reattivi}: il modello reagisce in maniera dinamica alle situazioni che si presentano: non `
prevista quindi una politica di default, non si stabiliscono risorse a priori come nel caso precedente, fornendo un
comportamento più flessibile. I costi risultano essere variabili, a seconda della complessità del sistema (al crescere
degli attori, aumenta il numero necessario di lock/semafori da utilizzare, per esempio). Il costo può quindi essere
limitato a seconda delle esigenze. 
\end{itemize}
Esempio: una serie di clienti vogliono accedere in maniera univoca ad una risorsa unica: si tratta quindi di un problema
di sincronizzazione e mutua esclusione.
Si potrebbe risolvere fornendo dei quanti di tempo decisi a priori per ogni singolo cliente, oppure si fornisce un
token, per cui chi lo possiede è l'unico a poter accedere alla risorsa.

È da osservare che un sistema distribuito in generale offre un insieme di politiche diverse, che si possono mescolare a
seconda delle esigenze: un sistema moderno infatti deve essere in grado di seguire l'evoluzione dei servizi per poter
sopravvivere (es CORBA: in versioni successive aggiunge supporto al Web).
Un ulteriore modello semplice da descrivere riguarda in che modo le risorse
sono associate alle applicazioni:
$\bullet$ Siamo in monoutenza se le risorse sono tutte dedicate all'applicazione
descritta
$\bullet$ Altrimenti si parla di multiutenza: più complessa, ma decisamente più
u
u
diusa. Infatti, un numero superiore di applicazioni può portare ad un
o
uso più eciente delle risorse.
u
A seguito, si può ragionare quindi come sono disponibili le risorse:
o
$\bullet$ Modello workstation: le risorse sono concentrate su un unico nodo.
$\bullet$ Modello processor pool : le risorse sono distribuite, e vengono associate in
maniera trasparente al servizio.
6
\subsection{Processi o oggetti?}
Un'ulteriore distinzione riguarda su cosa basarsi per descrivere eettivamente
come avviene l'esecuzione.
Un primo modello si basa sull'idea del processo, che incarna la capacità di
a
esecuzione della macchina. Si specicano quindi delle operazioni, e si tratta di
modelli che tengono conto della macchina su cui si lavora, e funzionano molto
bene con risorse locali e stato locali. Vi sono più modelli:
u
$\bullet$ Processi alla Unix: si tratta di processi pesanti, ognuno dotato di un
proprio stato locale. In questo modello interessa solo in che modo i processi
possono comunicare, essendo lo stato non condivisibile.
$\bullet$ Processi alla Java: i processi son leggeri, hanno uno stato condiviso che
può essere usato anche per comunicare. Risulta essere più facile fare anche
o
u
dei cambi di contesto, ma si possono presentare delle interferenze.
I thread di una JVM, proprio perché dotati di uno stato condiviso, si possono
e
rapportare con un'altra JVM. I thread leggeri utilizzano lo stato comune per
comunicare.
`
Un'altra idea riguarda la modellazione basata sul paradigma ad oggetti. E
importante ricordarsi che questi son sempre delle astrazioni, e quindi necessitano
di una concretizzazione successiva. Anche qui si possono supporre due diversi
modelli:
$\bullet$ Oggetti passivi : i classici oggetti alla Java. Rappresentazione l'astrazione
di dati su cui delle entità esterne possono lavorare. Non sono quindi
a
dotati di capacità propria d'esecuzione. Si può quindi avere concorrenza
a
o
fra diversi processi che vogliono accedere allo stesso oggetto, non si ha un
buon connamento e si ha scarsa protezione.
$\bullet$ Oggetti attivi : sono oggetti dotati di propria capacità d'esecuzione. I
a
processi quindi non entrano nell'oggetto, perché vi è già all'interno dele
e a
l'oggetto una propria logica di gestione (si ha quindi un modello a request/response: un processo richiede di poter
usare l'oggetto mediante i
{`}metodi' forniti da questo, e questo in base alla sua schedulazione risponde
`
alla richiesta E sempre da osservare che un oggetto attivo ha comunque
un proprio stato interno, descritto da una parte sequenziale, ma a cui
non vi si accede con dei metodi, proprio grazie a questa logica interna: i
metodi veri e propri sono utilizzati solo internamente!). Un oggetto attivo
` quindi dotato di una coda delle richieste che si sono presentate, e una
e
sua politica di scheduling. Si supera in questo modo il problema degli
oggetti passivi, proteggendo del tutto l'oggetto e determinandolo completamente. Un oggetto attivo quindi, massimizzando
la parte sequenziale,
si può spostare più facilmente. Rappresent lo stesso modello dei servitori
o
u
paralleli, e, per esempio, degli agenti mobili. Un oggetto attivo quindi
7
si preoccupa di tutta la gestione dell'esecuzione internamente (richieste,
attività, errori, processi interni. . . :si forniscono meccanismi, mediante il
a
quale si possono specicare delle politiche diverse!
Come mai in Java tale modello non è stato implementato? Si tratta di un mode
ello molto costoso, che non presenta delle scorciatoie utili se si lavorasse solo in
ambito locale!
Nei sistemi moderni, in realtà si parla di classi che contengono le denizioni
a
degli oggetti, e deniscono lo stato interno e i metodi con cui accedere. In particolare, in generale si parla di
semantica per riferimento, che però è in grado
oe
di funzionare solo in un ambito locale. Per riferire in remoto, serve un supporto
esterno. Per esempio, si parla di utilizzare socket, protocolli standard de facto
come TCP/IP. . .
Tuttavia, è proprio questo il motivo per cui si vengono a definire dei mide
dleware, ovvero delle strutture in grado di fornire un supporto completo alla
comunicazione in remoto. Per esempio, Java Remot Method Invocation è un
e
sistema in grado di estendere la semantica per riferimento locale, al remoto,
cercando di fornire un sistema omogeneo per poter accedere sia ad oggetti locali
che remoti. Nel caso specico di RMI, per esempio, si utilizza un classico pattern dei middleware, il proxy come delegato
per la comunicazione. Si realizzano
infatti uno stub lato cliente e uno skeleton lato server, che riescono a garantire
un qualcosa di simile alla normale comunicazione locale. Si cerca di presentare
un sistema trasparente per l'utente!
`
Perch` quindi non realizzare tutti i riferimenti come riferimenti remoti? E
e
sempre una questione di costo, sarebbe inaccettabile. Basta pensare a cosa
succede ad un oggetto passato via RMI: deve essere serializzabile, e così tutti
\i{}
i suoi componenti, in maniera che si possa fare marshalling e unmarshalling.
Ciò non è sempre possibile in Java RMI (si pensi ad un oggetto che riferisce un
o
e
oggetto di tipo File, una risorsa strettamente locale). Vi sono poi altri problemi
da considerare:
$\bullet$ Un oggetto remoto viene registrato in un apposito registry, ma se viene
deattivato? Come viene gestita quindi la persistenza? E lo stato remoto?
$\bullet$ Due stub diversi possono riferire lo stesso oggetto remoto? Se s` come si
\i{},
gestisce la concorrenza?
$\bullet$ Come risolvere i nomi? Si può usare un riferimento sso, ma l'oggetto può
o
o
essere diverso (ma non è il caso di Java).
e
$\bullet$ Si deve anche passare l'indicatore della classe, perché si possa realmente
e
utilizzare l'oggetto: cosa succede se la classe è già presente? Java per
e a
esempio controlla mediante l'hash se la classe è effettivamente quella core
retta.
8
\subsection{Deployment dell'applicazione}

Un'applicazione distribuita spesso necessita che sia suddivisa in diversi componenti, anche allocati su nodi diversi! Un
componente è un'entità con una
e
a
granularit` superiore all'oggetto, ovvero introduce comportamento e interazione,
a
estendo l'idea dell'oggetto.
Se si hanno quindi diversi componenti, che devono dialogare fra di loro, non
` detto che il deployment su una singola macchina sia la soluzione ideale: infate
ti, su un solo processore si ha una sequenzializzazione delle operazioni, e quindi
dell'applicazione stessa.
Spesso e volentieri, un middleware non si occupa del deployment, ma deve
essere realizzato da chi gestisce l'interazione fra i componenti. L'allocazione
può essere statica, cioè decisa prima dell'esecuzione, oppure dinamica, quindi
o
e
durante; quest'ultimo modello permette di poter spostare le risorse a seconda
delle esigenze durante l'esecuzione. Spesso e volentieri, l'allocazione viene realizzata utilizzando dei semplici le
batch, rendendola semi-manuale (certe congurazioni devono poi essere sistemate a seconda dei casi dall'operatore).
L'approccio misto è quello più usato, essendo una via di mezzo fra l'allocazione esplicita
e
u
(tutta a carico dell'utente) e quella implicita (decisa in toto dal sistema).
\subsection{Altri modelli oltre il C/S}
Esistono diversi altri modelli oltre il classico C/S:
$\bullet$ Push: il servitore fornisce il servizio al cliente
$\bullet$ Pull: il cliente recupera il servizio che necessita in maniera diretta.
$\bullet$ Modello a delega: il cliente delega un altro agente per attendere il risultato,
e poi lo recupera da questo.
$\bullet$ Modello a notica: simile al precedente, solo che il delegato notica al
cliente la disponibilità del risultato.
a
$\bullet$ Modello ad eventi: il tipico modello consumer/provider, in cui chi è intere
essato si registra a chi produce gli eventi.
$\bullet$ Modello a provisioning: oltre agli end-point, vi sono intermediari interessati al risultato.
Di particolare interesse è la classicazione dei modelli a scambio di messaggi :
e
$\bullet$ In fase di progettazione, si parla di sistemi sincroni se si è interessati al
e
risultati, altrimenti asincroni.
$\bullet$ In fase implementativa, si parla di comunicazione bloccante se il richiedente
si blocca in attesa del risultato, altrimenti non bloccante
9
Oltre alla classicazioni principali, altre caratteristiche con cui si possono classicare i modelli a scambio di
messaggi:
$\bullet$ In fase di progettazione si parla di
-- un sistema simmetrico se mittente e ricevente si conoscono (rarissimo) oppure asimmetrico (caso tipico:C/S).
-- un sistem diretto (cioè se avviene una comunicazione diretta fra i
e
processi), oppure indiretto (se si utilizzano strutture tipo le socket)
$\bullet$ In fase di implemtazione si parla di
-- un sistema buerizzato se la comunicazione richiede un certo numero
di messaggi prima di poter trasmettere
-- Sistemi reliable se possono fornire QoS, senza perdere messaggi.
Nel caso si introduca un terzo oggetto che faccia da mediatore, si può quindi
o
lavorare in maniera sincrona ma non bloccante: è il mediatore che resta in
e
attesa, sganciando il ricevente. Si possono avere due diverse modalità:
a
1. Oggetti poll : è un contenitore per la risposta alla richiesta, che viene
e
periodicamente interrogato dal ricevente con un sistema a polling. Si ha
così una comunicazione a 3 entità. In questo modello, il ricevente deve
\i{}
a
sapere dove si trova l'oggetto poll, e vi deve essere un oggetto per ogni
`
risultato. E conveniente se si hanno operazioni corte, e attese brevi.
2. Oggetti callback : è sempre un oggetto che conterr` il risultato, ma dotato
e
a
di una sua vita indipendente, e dotato di codice particolare da eseguire
`
quando giunge il risultato. E lui quindi a dover fornire il risultato al
richiedente, svincolandolo del tutto da questo! Conveniente in caso di
operazioni lunghe, indipendenti dal chiamante.
Spesso e volentieri, questi intermediari vengono realizzati da proxy, che sono in
grado di ridurre la complessità logica della soluzione, integrando diverse funzionalità.
In un modello basato sugli eventi, si ha un fortissimo disaccoppiamento (non
vuol dire che non si conoscono, ma che non è necessario che siano attivi conteme
poraneamente per poter fare la comunicazione; è una caratteristica fondamene
tale per i middleware) fra le entità interessate al risultato, e quelle che forniscono
a
i servizi. In questo modello si ha quindi un'interazione molti a molti, per cui si
può facilmente mappare il multicast. In generale gli eventi non sono persistenti,
o
tuttavia non è l'unico modello possibile.
e
10
\subsection{Spazi di tuple}
Un modello che deriva dall'astrazione della memoria condivisa e della comunicazione, e che garantisce il
disaccoppiamento fra le varie entità è lo spazio di
a e
`
tuple. E un meccanismo generale per la comunicazione e la sincronizzazione, in
cui si specica uno spazio di informazioni fortemente tipato. (Una tipica informazione può essere $<$ mittente,
destinatario, data $>$, per cui tutti i dati devono
o
essere diversi perché possa essere inserita)
e
Si hanno quindi due operazioni principali: la out consiste nell'inserire nello
spazio delle tuple un nuovo messaggio, mentre la in (bloccante) estrae la prima
tupla che fa match (in caso di più tuple che facciano match, è una scelta non
u
e
deterministica). Ovviamente, si avrà sempre un numero inferiore di in rispetto
a
alle out. Se la in è completamente specicata, si parla di sincronizzazione e non
e
di comunicazione. Uno spazio di tuple permette anche l'introduzione di QoS,
grazie alla persistenza della tupla nello spazio.
Gerlenter, con Linda, presenta due diversi modelli:
$\bullet$ A ring: si partiziona lo spazio delle tuple, per cui ogni nodo conosce
precedente e successivo: si ha una conoscenza locale, quindi limitata. Un
produttore può fare una out su un qualsiasi nodo, e il messaggio fa un
o
ciclo per vericare che non sia già presente su un qualche nodo. Lo stesso
a
discorso vale per la in, che fa un giro no a trovare un messaggio che fa
match. In Linda sono state considerate ovviamente delle problematiche
di replicazione (il messaggio deve essere ripetuto? Ma quindi lo si deve
togliere da tutti i nodi che lo contengono), sfruttando gli hash dei messaggi
$\bullet$ A matrice: le out vengono eseguite sulle righe, mentre le in si fanno sulle
colonne.
\subsection{Modelli a contenimento}
Si tratta di supporti presenti in ogni middleware: sono funzionalità che aga
gregano diverse attività utili, che si possono replicare. In questo modo, uno
a
sviluppatore si concentra soltanto sullo sviluppo della business logic vera e
propria, utilizzando le politiche di default fornite dal middleware per quanto
riguarda il supporto al tempo di vita, il gestore della concorrenza, fornire la
QoS, sistemi di nomi. . . Fondamentalmente, un container è un delegato che si
e
comporta come supervisore di certe attività.
a
\subsection{Il problema della trasparenza}
Esistono diversi signicati per la trasparenza: per esempio, si intende per l'allocazione l'indipendenza dalla località
delle risorse. In generale, per trasparenza si
a
intende un sistema in grado di fornire all'utente nale un comportamento omogeneo, anche se ai livelli sottostanti sta
lavorando con oggetti/servizi diversi
(remoti/locali, server principale/copia, . . . ).
11
Non è sempre un elemento positivo, perché in molti casi i sistemi son così
e
e
\i{}
complessi, che non si possono gestire in maniera totalmente trasparente: si
richiede l'intervento dell'utente. Un esempio è nella trasparenza dell'allocazione:
e
se non sappiamo dove si trova l'utente, come facciamo a dargli il servizio desiderato? La trasparenza non è corretta a
livello di infrastruttura in questo caso;
e
l'utente infatti è sempre più coinvolto.
e
u
Per esempio, TINA-C2 prevede sia un sistema trasparente, che un sistema
non trasparente in cui si possano definire le risorse inizialmente, e quindi introdurre la trasparenza.
\subsection{Le macchine astratte}
Son stati sviluppati diversi modelli per l'esecuzione, ma alla base vi è la Rane
dom Access Machine: è una macchina special purpose, con codice inalterabile,
e
in grado di eseguire istruzioni in sequenza (tutte le istruzioni hanno la stessa
durata), memoria limitata e un sistema di input e uno di output.
Un primo modello per la comunicazione è la Parallel RAM, in cui simmagina
e
diversi programmi da eseguire contemporaneamente. Una PRAM è costituita
e
quindi da P macchina RAM, ognuna delle quali esegue un programma, e ad ogni
clock vengono eseguite P istruzioni. Alla base vi è una memoria condivisa, per
e
tutte le macchine, dove scrivere e leggere i risultati (anche i dispositivi di I/O
sono unici per tutte le macchine). Si tratta di un modello MIMD sincrono, per
cui quando tutti i programmi niscono, anche la macchina termina. Essendo
dotata di un'unica memoria, come devino essere strutturate le operazioni? La
concorrenza è un requisito dicile da realizzare, in confronto al mettere le opere
azioni in sequenza. Le diverse PRAM si possono quindi categorizzare a seconda
della tipologia delle operazioni (in generale si hanno letture concorrenti e scritture sequenziali; la concorrenza `
invece utilizzata per il sistema di I/O, per
e
limitarne il collo di bottiglia). Un PRAM è dicile da realizzare perché non `
e
e
e
scalabile.
Un secondo modello è il Message Passing RAM, che è sempre una collezione
e
e
di RAM, ma ognuna con la sua memoria dedicata (i sistemi di I/O son sempre
condivisi). Si ha quindi che le diverse macchine comunicano mediante dei canali
prestabiliti (vi devono essere almeno P-1 connessioni). Si hanno istruzioni ad
hoc, cioè delle send e receive con rendez-vous, una semantica sincrona: questo
e
vuol dire che un'operazione sblocca la sua duale. Una MP-RAM è più facilmente
e u
realizzabile, e rappresenta meglio un modello locale. Se per esempio si volesse
eseguire un broadcast, servono almeno P-1 istruzioni, mentre una PRAM, per
quanto più dicile da realizzare, richiederebbe una sola operazione: è un modu
e
ello con maggiore capacità espressiva. Restano comunque entrambi dei modelli
a
troppo astratti per poter essere implementati realmente. Aggiungendo vincoli
2 Si tratta di un consorzio di aziende per le telecomunicazioni, con l'obiettivo di andare
oltre OSI: l'idea è quella di immaginare anche i possibili servizi, fornendo un modello più
e
u
coordinato, e un'infrastruttura con molti provider: si denisce un accordo/negoziazione fra i
fornitori dei servizi e i clienti, per poter per esempio stabilire la QoS, e quindi si fornisce il
servizio con i parametri indicati
12
e limitandoci quindi ad un'ipotesi di località, il progetto risulta più facilmente
a
u
realizzabile.
\subsection{Indicatori dell'efficienza}
In generale, ci si limita alla complessità temporale, TP (N ), dove P è il numero
a
e
di processori considerati: se è pari ad 1, siamo in un ambiente sequenziale.
e
Un primo indicatore è lo speed-up, che indica l'incremento di prestazioni se si
e
introducessero P processori:
SP (N ) =
T1 (N )
$>$1
TP (N )
(1)
Tuttavia, bisogna sempre tener presente che non tutti i programmi si possono
parallelizzare: ciò dipende se vi sono troppe dipendenze fra le sotto parti.
o
Un altro indicatore è l'efficienza nell'uso delle risorse:
e
SP (N )
P
(2)
T1 (N ) 1
·
TP (N ) P
(3)
EP (N ) =
Ma quindi, ciò corrisponde a dire:
o
EP (N ) =
Possiamo quindi definire lo speed-up massimo:
SP (N ) = P
(4)
EP (N ) = 1
(5)
E l'efficienza massima:
Sotto queste condizioni si avrebbe che tutti i processori lavorano al massimo,
ovvero non abbiamo processori idle che possono risultare inutili. Questi sono
coecienti statistici, validi per tutto l'algoritmo. Quello che si può osservare `
o
e
che la situazione ideale la si avrebbe se lo speed-up dipendesse in maniera lineare dal numero dei processori. Una
soluzione ideale non interessante sarebbe
se si distribuisse i compiti ai processi in maniera uguale (un programma totalmente parallelo!), raccogliendo solo alla
ne il risultato. In realtà, i casi più
a
u
interessanti sono quelli dove si presentano delle dipendenze, per cui necessariamente vi deve essere una parte
sequenziale: questo caso presenta quindi una
comunicazione intrinseca, indipendente dal deployment che si viene a realizzare.
Si possono quindi studiare diversi casi basandosi sui fattori di interesse N e
P . In particolare, si può definire come fattore di carico, o loading factor:
o
N
(6)
P
che denisce la complessità come viene suddivisa su ogni nodo. Vi possono
a
essere diversi casi:
L=
13
$\bullet$ N uguale a P : su ogni processore viene posta una parte molto semplice
del problema. Si dice ipotesi di identità.
a
$\bullet$ Dipendenza di N da P
$\bullet$ Indipendenza, interessante al crescere di N
La legge di Grosh stabilisce che la soluzione migliore risulta essere quella di lavorare sempre nel concentrato. . . Ma
ciò non è sempre possibile! Se per esempio
o
e
vi sono risorse limitate? Oppure, oltre un certo valore di N , non si riesce più
u
a lavorare nel concentrato, e la legge è sicuramente inutile nel caso di vincoli,
e
tipo la presenza di risorse distribuite.
La legge di Amdhal stabilisce che vi è un limite alla possibile parallelize
zazione, dato che ogni programma è costituito da una parte parallela e una
e
parte sequenziale, e quest'ultima limita lo speed-up: si ha quindi che non cresce
linearmente, ma tende a un valore asintotico! Nella situazione migliore immaginabile, con le risorse correttamente
allocate, si ha quindi che si avrà uno speeda
up sso, mentre sarà l'efficienza a variare a seconda dell'architettura del sistema!
a
Come calcolare quindi lo speed-up massimo? Si vuole avere il numero di
processori tali che si abbia la minore complessità del problema, realizzando un
a
sistema fortemente caricato: si denite quindi l'heavily loaded limit:
THL (N ) = infP (TP (N ))
(7)
Infatti, si lavora bene caricando molto ogni processore, i quali quindi lavorano
per tutto il tempo necessario: abbiamo quindi in questomodo un buon uso delle
risorse.
Tuttavia, per la legge di Amdhal, dobbiamo ricordare che vi è anche una
e
parte sequenziale in ogni programma. . . che è proprio la parte di comunicazione
e
fra i diversi processori! Al crescere del numero dei processori, si ha quindi
che lo speed-up diminuisce, allontanandosi dal valore massimo; questo degrado
delle prestazioni è dovuto proprio al peso crescente della comunicazione fra i
e
processori, che diventa fondamentale.
\subsection{Caso di studio: somma di N numeri}
Una possibile soluzione, imponendo la condizione d'identità, è quella di utilizzare
a e
un albero binario per fare la sommma: le radici sommano i numeri, che passano
il risultato al padre, e così via. Supponendo quindi un albero con profondit` H,
\i{}
a
possiamo osservare che N = 2H+1 e P = 2H+1 $-$ 1, per cui P è simile a N . Si
e
ha che quindi:
TP (N ) = O(H)
= O(log2 (N ))
= 2 · log2 (N )
14
Perch` 2? Perch` abbiamo 2 comunicazioni in ogni nodo. Possiamo quindi
e
e
valutare l'efficienza di questa architettura:
1
log2 (N )
EP (N ) = O
Ma quindi, al crescere di N, l'efficienza tende a 0! Per quanto riguarda lo speedup, si ha che non lavorano tutti i
nodi,
ovvero più si risale l'albero e meno i
u
processori lavorano (il processore radice deve attendere, ovvero tempo di idle,
che tutti gli altri abbiano nito!). Se invece avessimo un usso di dati, i processori risulterebbero tutti sempre
impegnati (potremmo risolvere un insieme di
problemi).
Come si possono quindi mantenere sempre impegnati i processori? Si deve
incrementare il lavoro sul singolo processore, per cui:
L=
N
$>$$>$ 1
P
(8)
e quindi la complessità deve essere superiore al numero dei processori: su ogni
a
nodo si ha quindi un certo numero di valori da sommare. Ciò si può ottenere
o
o
per esempio con un albero sso, ovvero un numero di processori ssi. Infatti:
TP (N ) = O (L + log2 (P ))
I due componenti rappresentano rispettivamente il tempo di computazione e
quello di comunicazione.
SP (N ) =
T1 (N )
TP (N )
=O
=O
N
P
N
+ log2 (P )
P
P
+
1
N · log2 (P )
Per cui, lo speed-up tende eettivamente a P
SP (N )
P
1
1
=O
+
1 N · log2 (P )
EP (N ) =
E l'efficienza tende ad 1. Quindi, caricando al massimo i processori ed usandoli
correttamente, si ottiene speed-up ed efficienza massimi.
\subsection{Considerare l'I/O}
In questi indicatori si è tuttavia lasciato da parte il problema dell'I/O, che
e
spesos è il vero collo di bottiglia in diverse architetture. Inoltre, vi son diversi
e
15
fattori che possono inuenzare l'efficienza e lo speed-up dell'architettura, come
il deployment reale. Si possono quindi estendere gli indicatori considerando
l'eetto generale dell'overhead, T0 (N ), che rappresenta le risorse e il tempo
realmente utilizzato per la comunicazione (caso ideale di deployment):
T0 (N ) = |T1 (N ) $-$ P · TP (N )|
(9)
Ma quindi, si ha che:
T0 (N ) + T1 (N )
P
E si ha che che lo speed-up risulta essere:
TP (N ) =
SP (N ) =
P · T1 (N )
T0 (N ) + T1 (N )
(10)
(11)
E che l'efficienza risulta essere:
EP (N ) =
1
1+
T0 (N )
T1 (N )
(12)
Per cui l'efficienza non sembra dipendere dal numero dei processori. . . in realtà,
a
` proprio T0 (N ) a dipendere dal numero dei processori : l'overhead dipende dal
e
numero di processori impegnati.
Supponiamo quindi di avere come obiettivo per l'architettura di mantenere
costante l'efficienza:
1
E=
1+
E+E·
T0 (N )
T1 (N )
T0 (N )
=1
T1 (N )
1$-$E
T0 (N )
=
T1 (N )
E
(1 $-$ E)
T0 (N ) =
· T1 (N )
E
T0 (N ) = K · T1 (N )
Se K fosse veramente una costante, saremmo in isoefficienza. Un sistema isoeciente indica che, se manteniamo costante la
complessità e aumentiamo il
a
numero dei processori, l'efficienza non varia. Il valore di K quindi determina se
il sistema ha un buon comportamento. In particolare:
$\bullet$ K piccolo: il sistema è altamente scalabile. Al crescere di K quindi la
e
scalabilità del sistema decresce
a
$\bullet$ Se non è costante ma funzione dei processori, allora il sistema non `
e
e
scalabile.
I sistemi reali sono scarsamente scalabili.
16
\subsection{Conclusioni}
Un progetto deve essere valutato attentamente, per cercare di dimensionarlo in maniera corretta: nel caso di una
macchina singola l'heavily loaded limit è un ragionamento corretto, ma bisogna ricordarsi che si deve sempre cercare di
parallelizzare (non utilizzare la legge di Grosh). Per esempio, qual è il numero ideale di processi da realizzare? Si
può ben pensare che avere meno processi di processori risulti in processori che non lavorano, e quindi in un sistema 
non efficiente: il limite superiore quindi è che il numero dei processi sia lo stesso del numero dei processori. Un
processore è idle quando comunica, quindi si dovrebbe limitare la comunicazione, e quindi troppi processi comunicano
troppo.
In realtà, statisticamente si ha che un processore che lavora con 20 processi presenta ancora lo stato di idle: un
centinaio di processi per processore è un numero adeguato.