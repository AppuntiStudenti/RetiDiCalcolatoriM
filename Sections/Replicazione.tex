\section{Modelli per la replicazione}
Un obiettivo per poter fornire della qualità di servizio è ovviamente quello di poter offrire un servizio continuativo,
stabile: se il servizio non viene erogato, non si è remunerati (anzi, in determinati ambiti è proprio un danno economico
perché si possono presentare anche delle richieste di risarcimento; in certi ambiti si parla anche di rischi umani).
L'obiettivo è quindi quello di realizzare un sistema \textit{fault tolerant}, resistente ai guasti. Sono invarianti che
devono essere comunque garantiti.
\subsection{Definire i comportamenti errati del sistema: cause ed effetti}
I possibili guasti si possono catalogare in maniera logica:
\begin{itemize}
 \item Si parla di \textit{failure} quando il sistema presenta un comportamento diverso dagli invarianti di sistema
previsti: ciò è dovuto a un sistema progettato male. Il failure è l'effetto visibile, riscontrabile dall'utente.
 \item Un errore è invece il difetto che genera un failure del sistema, la causa concreta che ha portato il sistema 
 in uno stato non corretto.
 \item Un fault infine è proprio il comportamento che si è venuto a vericare che ha portato il sistema a generare un
 failure: corrisponde quindi proprio alla causa a monte.
\end{itemize}
I fault si possono classicare a seconda di quanto si ripetono:
\begin{itemize}
 \item Sono \textit{permanenti} se si ripresenta in maniera periodica: sono quindi facili da individuare, e facili da
 risolvere. Si parla di Bohrbug per errori che si ripetono facilmente, riproducibili, che si possono quindi osservare
 e correggere facilmente.
 \item Se invece sono \textit{transienti}, sono più dicili da notare. Si parla di errori di tipo Eisenbug, e sono molto
 dicili da eliminare.
\end{itemize}

\subsection{Ipotizzare il guasto}
Esistono diversi parametri per poter giudicare la disponibilità di un servizio/sistema.
Un sistema è fault tolerant se garantisce la \textit{dependability}, ovvero si ha confidenza per qualunque aspetto
progettuale. Ciò comporta che il sistema debba garantire:
\begin{itemize}
 \item \textit{Reliability}: deve essere affidabile, in ogni modo il sistema si comporta correttamente rispetto agli
invarianti che sono stati imposti. È quindi un sistema corretto.
 \item \textit{Availability}: indica la disponibilità nel tempo del sistema. La risposta da parte del sistema deve
arrivare entro una certa deadline prestabilita.
\end{itemize}
Per poter garantire la disponibilità del sistema a fronte di guasti, si devono fare delle ipotesi sul guasto. In
generale, la procedura di ripristino consiste infatti prima nell'identicazione del tipo di guasto, e in base a ciò si
cerca di riattivare con la corretta procedura il servizio (fase di recovery).
In generale, si utilizza l'ipotesi del \textit{fault singolo}, ovvero nel sistema si può presentare un singolo guasto:
solo quando il sistema torna operativo, si ha che si potrebbe presentare un ulteriore fault. Quest'ipotesi è necessaria
per poter semplicare il trattamento del recovery. Per poter definire ciò, è necessario conoscere:
\begin{itemize}
 \item \textit{Il Time To Repair}, TTR, che è il tempo necessario per accorgersi e per poter correggere il guasto
 \item \textit{Il Time Between Failure}, ovvero il tempo che intercorre fra due failure (oppure la media, MTBF).
\end{itemize}

L'obiettivo è quindi che il TTR sia inferiore in maniera stretta al (M)TBF.
Mediante l'ipotesi del guasto singolo, si può osservare che se abbiamo più copie in grado di fornire il servizio, si
può specificare il numero di fault tollerabili e identicabili. Se avessimo solo due copie, non possiamo tollerare un
guasto, ma possiamo identificarne uno. Con già 3 copie, un guasto si può tollerare,
perché comunque restano due macchine con cui identificare i guasti successivi:
si riescono così ad identificare no 2 guasti. La generalizzazione è che se si hanno 3t copie, si tollerano massimo t
guasti.\footnote{Non si fanno ipotesi sul tipo di guasto, ma solo sul fatto che è singolo} Si possono quindi
ridefinire le proprietà precedenti in base a questi valori:
\begin{itemize}
 \item La reliability è il valor medio sulla disponibilità/indisponibilità della risorsa considerata
 \item L'availability si può esprimere come la percentuale utile di lavoro che il sistema riesce ad offrire:
 \begin{equation}
  A = \frac{MTBF}{MTBF + MTTR}
 \end{equation}
\end{itemize}
In particolare, si possono distinguere i casi di lettura e scrittura: infatti, la lettura non modifica lo stato, per
cui si può fornire un servizio di lettura anche se vi sono diverse copie non funzionanti. Viceversa, per garantire uno
stato coeso fra tutte le copie, la scrittura richiede che tutte siano disponibili.
La reliability si può anche vedere come la probabilità che un servizio sia disponibile per un certo intervallo di 
tempo (a 0 deve corrispondere con l'availability).
Esistono altre proprietà fondamentali, come la correttezza e la vitalità: la prima definisce che comunque sia, il
risultato fornito dal sistema sarà corretto, la seconda invece stabilisce che comunque si raggiungerà l'obiettivo.
L'ideale sarebbe disporre di entrambe le proprietà, così da poter mascherare i fault: si può ottenere mediante 
oppurtune tecniche di replicazione spaziali e/o temporali:
\begin{itemize}
 \item Se si hanno diverse macchine, che eseguono ognuna un algoritmo diverso ma forniscono lo stesso risultato,
 garantiamo la massima correttezza
 \item Se si hanno invece diverse macchine ognuna con un compito specifico, si ottimizzalo throughput del sistema
\end{itemize}
Una singola macchina non basta per \textit{identificare e correggere} un guasto! La replicazione è fondamentale,
almeno per fare monitoring (utilizzo di cluster per fare il controllo di una risorsa): si potrebbero utilizzare delle
speciche parti per controllare e correggere l'architettura, ma è necessario rispettare sempre il principio della minima
intrusione: il rischio sarebbe che troppe risorse sono allocate solo per vericare che il sistema funzioni correttamente,
riducendone l'efficienza! La replicazione è un costo che si va ad aggiungere al sistema: ma in generale si tratta di un
costo sso, a fronte di possibili costi molto peggiori in caso di fallimento del sistema!
\subsection{Possibili guasti}
I guasti si possono anche classicare in base alla loro riconoscibilità da un sistema (processore esterno):
\begin{itemize}
 \item \textit{Fail-stop}: un processore si ferma perché non rispetta un invariante, e questo viene identicato dagli
 altri processori
 \item \textit{Fail-safe}: come il precedente, tranne che gli altri processori non se ne accorgono: in questo caso
 quindi non è identicato, ma potrebbe essere tollerato (dipende dall'architettura stessa).
 \item Fallimento di tipo bizantino: il processore termina presentando però comportamenti assolutamente casuali.
\end{itemize}
Nelle reti di calcolatori si possono poi presentare ulteriori generi di guasti, dovuti alla mancata totale
ricezione/trasmissione di messaggi (\textit{send/receive omissions}), oppure anche al fatto che un processore si 
blocca. Le reti poi possono aggiungere problemi (se un router per esempio non rispodne più, si potrebbe avere che la
rete risulta partizionata in due sottoreti, incapaci di comunicare fra di loro).
\subsection{Architetture per garantire la fault-tolerance}
Nel corso del tempo sono state pensate e realizzate diverse tipologie di architetture in grado di fornire questo genere
di servizio.

Una prima ipotesi operativa è quella di realizzare un sistema con vera e propria replicazione hardware, per cui si 
hanno due possibilità:
\begin{enumerate}
 \item O si realizza un'architettura in cui vi è una sola macchina a lavorare, mentre un'altra ne controlla il corretto
 funzionamento, per cui interviene solo in caso di guasto. Per esempio, per avere la correttezza si potrebbe
 avere che a una determinata richiesta rispondano tutte le macchine, ma la risposta viene controllata prima di essere
 fornita.
 \item Oppure si realizza un sistema con cluster, dove si aggiunge una logica di controllo che è in grado di stabilire
  se una risorsa è disponibile o meno, e quindi è in grado di escluderla dal sistema. Al crescere delle risorse
  aumenta la probabilità di guasti, quindi si devono utilizzare algoritmi efficienti per identicarli correttamente: si
  vuole un \textit{metalivello efficiente}!
\end{enumerate}

Un'ipotesi spesso comune è quella della memoria stabile\footnote{Si potrebbe realizzare mediante la tecnologia RAID,
Redundant Array of Inexpensive Disks; è un sistema a basso costo, inizialmente realizzato per poter velocizzare la
lettura e quindi per poter fornire un semplice supporto alla replicazione. Le operazioni sui dischi son quindi
coordinate. In realtà, quando un disco si guastava, essendoci problemi nel sistema di gestione non si aveva mai una
replicazione veramente buona}: si vuole realizzare un sistema per cui la memoria persistente dell'architettura non
fallisca mai (sempre disponibile, sempre corretta). Una possibilità per realizzare ciò consiste nell'utilizzare due
dischi, e fare in modo che ogni blocco sia replicato in maniera uguale su entrambi, con probabilità d'errore congiunta
nulla (caso dell'errore singolo). Si ha che quindi i singoli blocchi possono indicare la presenza di errori o meno
(omissions), con l'uso di appositi codici di controllo. Mediante questi indicatori infatti si ha il controllo se le
copie sono uguali o meno:
\begin{enumerate}
 \item Sulla seconda copia potrei avere dei valori non corretti
 \item Sulla seconda copia potrei avere dei valori corretti ma diversi dalla prima copia
\end{enumerate}
Il secondo caso in realtà è il peggiore, dovuto magari a problemi di aggiornamento sulla seconda
copia\footnote{Osservazione: è sicuramente conveniente un'indicazione di tempo per la scrittura, in maniera tale da
poter aggiornare tutte le copie al dato più recente effettivamente presente}: clienti che accedono alla seconda copia
prima della fase di recovery non potrebbero accorgersi che il dato è stale. Ogni operazione deve procedere su tutte le
copie quindi per poter garantire risultati corretti! Tale sistema è sicuramente costoso e dicile da realizzare.
Un'ulteriore aspetto riguarda un sistema software per fare il monitoraggio delle risorse, necessario e conveniente da
mettere in parallelo al sistema di recovery hardware: si tratta quindi di progettare degli appositi protocolli e sistemi
in grado di permettere all'applicazione di ripartire con il minimo costo e nel minimo tempo. Il problema di ciò è che un
sistema software del genere sfrutta le stesse risorse dell'applicazione (CPU, RAM...), riducendone l'efficienza: si
deve quindi progettare il tutto rispettando il \textit{principio della minima intrusione}, limitando il più possibile
quindi le risorse allocate al metalivello di monitoring/supporto dell'applicazione. La replicazione presenta dei costi
elevati in termine di realizzazione, ma anche di progettazione ed uso delle risorse, quindi deve essere attentamente
studiata. Ipotizzare guasti singoli semplica la realizzazione dei protocolli, per esempio.

Una possibile realizzazione è un sistema tandem, cioè in cui tutto è raddoppiato: l'idea è quella di realizzare un
sistema fail-safe, per cui una CPU identica l'errore. Ovviamente, un sistema del genere è complesso e costoso.
\subsection{Copie calde e copie fredde}
Vi sono due possibili modelli per la replicazione, e dipendono dal comportamento delle copie:
\begin{itemize}
 \item \textit{Copie fredde}: vi è un'unica copia attiva che funziona, più diverse copie dormienti. Deve essere quindi
 presente anche un manager del sistema, in grado di attivare in caso di necessità una delle copie dormienti (ovvero se
 la copia principale presenta un fault e smette di lavorare correttamente). Il manager si preoccupa quindi di attivare
 una nuova istanza dell'oggetto solo quando la precedente non funziona. Questo sistema presenta un alto tempo di
 configurazione, e ha il problema che lo stato non è direttamente salvato sugli oggetti in standby (si dovrebbe
 recuperare in un qualche altro modo...).
 \item \textit{Copie calde}: in questo caso vi sono diversi oggetti, pronti a sostituire una copia che non funziona:
 l'idea è che in questo caso vi sia invece un protocollo di ruolo, in grado di indicare ad ogni copia che comportamento
 deve assumere (è la principale o meno): infatti, la copia principale è attiva, ovvero in grado di salvare lo stato,
 mentre le altre son passive, che vengono sempre aggiornate ad ogni modica sulla copia attiva. Se fallissero anche 
 tutte le copie calde, si può ripiegare su un sistema a copie fredde, in grado di inizializzare una nuova copia.
\end{itemize}
\subsection{Gestione delle risorse}
La replicazione delle risorse è necessaria per un sistema distribuito, perché introducendola si può garantire la
fruizione del servizio. In un progetto quindi si deve considerare come allocare le risorse sui vari nodi, e con che
grado di replicazione: risorse replicate signica quindi che si hanno copie multiple delle risorse su nodi diversi!
Maggiore è l'importanza della risorsa, maggiore deve essere il suo grado di replicazione.
Si possono ipotizzare sempre due modelli base:
\begin{itemize}
 \item \textit{Modello passivo}: una sola risorsa lavora, e le altre sono in attesa pronte a sostituirla in caso di
 failure. Il modello generale è quindi quello di un master con una serie di slave, organizzati in maniera gerarchica.
 Se il numero di partecipanti è limitato non è costoso, ed è il modello più facile da implementare (ha un protocollo
 chiaro e semplice). Il grado di replicazione è quindi indicato dal numero di copie passive disponibili.
 \item \textit{Modello attivo}: tutte le copie eseguono assieme per ottenere il risultato desiderato. Sono quindi tutte
 attive, ed è necessario predisporre una parte di controllo abbastanza complessa, quindi la modellazione e
 l'implementazione sono più costose (tutte le copie devono avere l'input, e tutti gli output devono essere confrontati).
\end{itemize}
In entrambi i modelli non si lavora in maniera sequenziale, ma parallela: le copie possono comunque svolgere anche 
altri compiti, eseguire più operazioni contemporaneamente!

Nel primo modello si deve pensare ad un sistema per tenere aggiornate le copie fredde: si utilizzano quindi dei
checkpoint, per cui lo stato viene trasferito e replicato sulle copie fredde... ma quando farlo?
\begin{itemize}
 \item Se lo si fa prima di fornire la risposta al cliente, si garantisce un'alta efficienza a scapito della 
 correttezza (se gli slave presentano un guasto, non saranno aggiornati correttamente)
 \item Se invece si attende che tutte le copie garantiscano di aver ricevuto l'aggiornamento prima di spedire la
 risposta, il sistema è corretto, ma la risposta avrà un tempo molto lungo.
\end{itemize}
L'aggiornamento può essere eseguito periodicamente (\textit{time-driven}) oppure ad ogni volta che si presenta un nuovo
evento (\textit{event driven}). Quest'ultima politica risulta essere maggiormente dinamica e complessa da realizzare.
Per esempio, un evento potrebbe essere scatenato dall'inizio del servizio della richiesta (estrazione dalla coda delle
richieste), e terminare alla fine per poter fare il checkpoint.

Le operazioni poi possono correlate o meno, e quindi anche il checkpoint potrebbe essere correlato! In questo caso,
servono delle informazioni dall'utente su quando conviene fare il checkpoint (tipicamente, si attende che si giunga ad
uno stato stabile e poi lo si esegue).

Serve quindi la trasparenza? Il cliente deve sapere chi contattare se il master non è raggiungibile, quindi no! Deve
avere idea che non sta interagendo con la copia principale, per poter sfruttare le risorse per potersi collegare alla
risorsa secondaria mediante un opportuno sistema di nomi. Inoltre, a seconda della complessità del master, potrebbe
essere conveniente farlo ritornare direttamente come master, invece che come slave. Si può comunque realizzare così una
sorta di \textit{fault-transparency}, per cui dall'esterno la risorsa sembra reggere diversi fault e riesce a fornire
un servizio con continuità.

Infatti, sarebbe meglio che all'esterno non si sapesse il grado di replicazione di una risorsa, perché altrimenti
sarebbe una decisione distribuita mediante il deployment: è in realtà il middleware ad accorgersi e a fare da supporto,
contattando la nuova copia: per il client è tutto trasparente!

Nel caso di copie attive, il client come le deve conoscere? Se le conosce in maniera esplicita, non abbiamo trasparenza
e fornisce visibilità alla replicazione: viene utilizzato solo in appositi sistemi ad-hoc. In realtà, vi è sempre un
sistema implicito a garantire la comunicazione fra il client e le copie attive. E quindi necessario pensare ad un
frontend che si preoccupi di smistare opportunatamente le richieste: potrebbe essere statico (in grado quindi di
smistare qualunque richiesta) oppure dinamico (specializzato per una determinata richiesta). Ma nel caso di oggetto
singolo, questo diventa un collo di bottiglia, perché se viene a mancare l'architettura non si regge più in piedi! 
Deve \textit{garantire un'alta affidabilità}.

Il secondo modello è realizzabile facendo in modo che ogni client diventi il gestore di quel tipo di richiesta, ma 
ciò necessita che le varie copie si mettano d'accordo: vi è un problema di sincronismo. Si può risolvere in diversi
modi (sfruttando una politica ad anello, utilizzando un token univoco, in modo da simulare così un gestore unico fra
tutte le copie attive), ma utilizzando delle approssimazioni per la sincronia, si possono ottenere dei protocolli più
semplici da implementare (si possono presentare dei problemi dal punto di vista semantico, ovvero come devono essere
ordinate le operazioni), risultando meno costose e più veloci. Infatti, in caso di richieste indipendenti fra di loro,
la perfetta sincronia è soltanto un costo aggiuntivo! La sincronicità deve quindi essere studiata a seconda del tipo di
operazioni che si devono svolgere (letture sincrone, scritture sequenziali). Per altre operazioni, si deve tener conto
dell'architettura e del signicato dell'operazione: si potrebbero eseguire in maniera indipendente, e procedere quindi
successivamente ad una riconciliazione fra le copie per avere uno stato univoco coeso. E importante però nel caso di
copie attive che l'aggiornamento delle copie preceda la consegna del risultato al cliente.

E quindi necessario pensare ad un gestore delle copie attive, in grado di escludere correttamente le copie non
funzionanti (altrimenti si avrebbero tempi di attesa altissimi per avere conferma degli aggiornamenti; se un'operazione
si guasta, deve esserci comunque un modo per poter fornire una risposta all'esterno) e quindi di poter essere in grado
di coordinare le diverse copie attive. Per esempio, invece che utilizzare un sistema per cui tutte le copie garantiscono
la conferma dell'azione, si potrebbe utilizzare una forma di voting: solo la maggioranza delle copie in concordanza fra
di loro proseguono l'esecuzione, mentre le altre dovranno essere sospese e si dovrà procedere con il recovery
(identicazione e recupero dal gestore o da chi altro: dovrà essere deciso in fase progettuale).
Questo protocollo sgravia la risposta verso l'esterno, che può essere fornita più velocemente! E evidente la necessità
di un gestore per il monitoraggio, controllo delle copie. In generale, in un sistema reale, il numero di copie è
limitato: studiando opportunatamente il protocollo si riduce l'overhead al minimo.
\subsection{Modelli per la replicazione, copie attive}
L'ipotesi è che vi è un insieme di copie che devono produrre un risultato per un insieme di clienti. Si deve quindi
pensare a delle opportune fasi di coordinamento sia prima che dopo l'esecuzione, per poter organizzare le copie e
fornire il risultato deciso. Si possono quindi pensare modelli in cui le copie lavorano sì in maniera indipendente,
ma all'occorrenza sarebbero in grado di fare da supporto a copie non funzionanti, ottenendo così un bilanciamento del
carico.

Nel caso delle copie attive, vi sono 5 passaggi che si devono fare in ordine:
\begin{enumerate}
 \item All'arrivo di una richiesta, come deve essere smistata? Vi sono due possibilità: o arriva ad un'unica copia
 che poi deve fornirla anche alle altre, oppure viene mandata dal client direttamente a tutti. Per avere maggiore
 trasparenza, è ovviamente preferibile il primo modello. La copia quindi può avere un comportamento dinamico o statico.
 \item Vi è poi la fase di coordinamento fra le copie: dipende nuovamente dall'architettura che si vuole realizzare. Si
 potrebbe avere un unico master gestore, oppure tutte le copie possono eseguire o in maniera paritaria o sono pesate 
 a seconda della loro importanza. Questa fase può servire per bilanciare correttamente il carico fra le varie copie
 attive.
 \item Quindi vi è la vera esecuzione. In generale tutte le copie devono eseguire, ma non devono farlo in contemporanea.
 Tutte hanno stato, e devono fornire un risultato. Per certi servizi vi potrebbe essere una copia dedicata che può
 fornire subito la risposta.
 \item Seconda fase di coordinamento: il master (oppure le copie con un sistema di voting) determina il risultato
 finale basandosi sulle risposte ottenute da tutte le copie. Questa è anche la fase in cui si possono individuare i
 guasti. Questa fase può essere utile nel caso si fossero ricevute più richieste correlate fra di loro, e quindi si
 potrebbe evidenziare un ordine non corretto d'esecuzione. Ciò però non è sempre possibile, quindi bisogna pensare alla
 prima fase di coordinazione mediante l'uso di un'operazione atomica, così da evitare un eventuale undo. In base quindi
 all'ordine delle richieste, le varie copie sanno come eseguire.
 \item Presentazione del risultato al cliente, che deve essere univoco.
 Il protocollo è molto più complesso di quello a master/server, tuttavia per gradi di replicazione limitati, ovvero
 con poche copie e un protocollo definito in maniera efficiente si ha un tempo di risposta piccolo.
 Un'architettura basata su master/slave (copie passive) riduce di molto i costi, visto che vi è un'unica macchina a
 lavorare: produce il risultato e si preoccupa di fare l'operazione di checkpoint su tutte le altre copie, e quindi lo
 fornisce al cliente. La gestione è maggiormente semplificata.
\end{enumerate}
\subsection{Politiche di aggiornamento}
Esistono diverse possibilità per eettuare l'aggiornamento delle copie. Vi potrebbe
essere una copia primaria che si preoccupa di aggiornare le altre, oppure tutte
le copie possono assumere questo compito. Si deve poi decidere se privilegiare
la correttezza (situazione eager o pessimista, si prevede di aggiornare tutte le
copie prima di fornire la risposta al cliente, quindi a scapito della prontezza della risposta), oppure il tempo di
risposta (situazione lazy o ottimista, aggiorno
prima il client e poi le altre copie).
$\bullet$ Copia primaria eager : una copia esegue, si preoccupa di aggiornare tutte
le altre copie, e quindi fornisce il risultato al cliente. Esegue quindi
un'operazione alla volta
$\bullet$ Copia primaria lazy: aggiorna prima il cliente poi le altre copie. Pu`
`
quindi eseguire più operazioni alla volta. E compito del gestore controllare
gli aggiornamenti e vericare che siano stati recepiti da tutte le copie in
maniera corretta, coordinandole.
$\bullet$ Aggiornamento di tutte eager : tutte le copie devono eseguire, mettesi d'accordo (two-phase commit) e quindi
fornire il risultato: è in questo caso
che si potrebbero avere undo costosi. Un'altra soluzione sarebbe che una
copia, ricevuta la richiesta, la propaga a tutte utilizzando un'operazione
multicast atomica.
Il problema del coordinamento è che è costoso: tuttavia, è anche lo strumento
che ci fornisce le maggiori garanzie di correttezza. Vi possono essere dei rilassamenti, per esempio: si parla di copie
tiepide quando i checkpoint sono fatti a
determinati intervalli, per ridurre il costo dell'operazione.
25
\subsection{Clustering}
L'idea del cluster è quella di realizzare un insieme di risorse replicate sempre
disponibili (high availability), cercando di snellire il tutto per non avere un tempo di risposta troppo alto. L'idea
quindi sarebbe avere un protocollo semplice
e le soluzioni sempre disponibili (ma spesso è solo un bello slogan).
Per realizzare ciò si utilizzano componenti o the shelf, a basso costo e facilo
mente reperibili e sostituibili (alla Google). Si può cercare anche di bilanciare
il carico: all'arrivo di una richiesta, una sola copia lavora, possibilmente quella
più libera (esecuzione dinamica quindi). In generale quindi:
$\bullet$ Si riceve la richiesta
$\bullet$ La si smista alla macchina meno carica
$\bullet$ Esecuzione
$\bullet$ Viene fornito il risultato al client.
`
Per decidere quale copia debba eseguire, come si fa? E necessario un sistema
di monitoring (necessario per la QoS), uno scheduler (in generale strutturato
master/slave). Questo sistema deve anche essere in grado di identificare eventuali copie guaste, e quindi di escluderle
per mantenere delle buone prestazioni,
possibilmente facendo migrare i servizi sulle macchine al momento migliori. Per
determinare se una copia è attiva, si usano degli heartbeat, ovvero si mandano
dei messaggi per vericare se la copia è viva: se non risponde, si suppone che
ci siano problemi. Questo sistema deve essere correttamente dimensionato (ogni quanto si manda l'impulso? Qual'` il
tempo di comunicazione? Il massimo
ritardo accettabile? In generale vi è una macchina apposita per questo compito).
Nel caso di fail-over, vi sono due diverse politiche da attuare:
$\bullet$ O si attiva una macchina che era passiva
$\bullet$ Oppure si sfrutta una macchina che era già attiva per un altro servizio,
caricandola ulteriormente. Ciò può essere costoso alle volte
o o
Se l'heartbeat non è progettato correttamente, potrebbe causare fail-over!
Se si partizionano le risorse (abbiamo per esempio due sottoreti) cosa succede? Pu` essere problematico, il cluster
dovrebbe lavorare comunque! Si deve
ripristinare la rete globale. Il servizio deve funzionare anche se vi sono problemi
di coordinamento, quindi in realtà si prosegue lo stesso, e si rinvia l'aggiornaa
mento ad un secondo momento. Le due sotto-reti quindi si coordineranno più
avanti!
