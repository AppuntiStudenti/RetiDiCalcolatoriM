\section{CORBA}
CORBA è uno standard, specifica di un mw, nato dall'idea di mettere a dispoe
sizione di 440 aziende una possibile soluzione per le loro esigenze!
Essendo uno standard, quindi solo una specifica cartacea, si ha che esistono
diverse implementazioni (IBM, Java, Orbix, Jacorb, . . . ), ma che rispettando le
specifiche possono anche essere interoperabili fra di loro!
Un mw deve avere oggetti remoti, invocabili da diversi client su diverse macchine: il sistema di supporto deve essere
quindi in grado di ottenere tutti i servizi
conosciuti da parte del client. Si ha quindi una forte idea di razionalizzazione,
l'obiettivo è superare tutti i problemi dell'eterogeneità (in CORBA: supporto
per linguaggi ed ambienti diversi, denizione di interfacce per servizi e supporti
per gli oggetti, . . . ).
CORBA in particolare basala sua architettura sulla presenza di un broker,
detto Object Request Broker : è infatti l'incarnazione di tutta l'architettura,
perché fornisce il supporto alla comunicazione, il controllo degli oggetti, facilita
la comunicazione, ed alloca gli oggetti. . . ), ovvero è il bus per l'interconnese
sione!26 Una caratteristica importante degli ORB è che è prevista la possibilità
di coordinarli, grazie alla denizione di uno standard preciso (non dipende quindi dalle varie implementazioni!). In
questo modo, tutti i servizi riconosciuti da
un ORB saranno riconosciuti da un altro, magari introdotto in un secondo
momento27
CORBA è un mw maturo, che presenza diversi servizi: per esempio vi sono
le common facilities, che sono dei servizi utili per poter aiutare tutte le possibili
applicazioni, in grado di fornire risposte ad esigenze specifiche. Corrisponde in
tutto e per tutto al terzo livello dei mw.
I servizi veramente importanti, necessari e presenti in ogni implementazione
sono gli object (o CORBA) services, che permettono di fornire servizi utili per gli
oggetti (il loro trasporto, la funzione di narrowing, . . . ). Fra quelle fondamentali
vi sono:
$\bullet$ Servizio di nomi : rispetto a quello definito per Java RMI, è molto più coe
ordinato, e presenta anche delle varianti. Infatti, oltre al normale servizio
di nomi (per cui si deve conoscere il nome logico del servizio) vi è anche la
possibilità di definire un servizio di trading, che funziona come le pagine
gialle, ricercando servizi e non nomi!
$\bullet$ Event e notication: sono sistemi meno orientati agli oggetti, ma che
quindi aumentano le capacità del sistema! Infatti, oltre alla maniera sina
crona/asincrona bloccante, in questo modo si possono introdurre sistemi
molto più essibili.
26 Tuttavia, CORBA resta solo una specifica: come debba essere fatto il deployment
dell'ORB è in realtà a carico delle singole implementazioni
27 Questo servizio per esempio è più potente del servizio di nomi dei WS, che è solo locale
e u
62
Altri servizi possono essere specici a determinati domini lavorativi, altri potrebbero invece dipendere dal tipo di
applicazione che si sta sviluppando.
Una caratteristica che è stata sempre trascurata è la sicurezza: questa però è introducibile, visto che CORBA prevede 
la possibilità di utilizzare degli interceptor!
CORBA è definito da una serie di componenti: oltre all'ORB (uno dei
principali), vi sono:
$\bullet$ L'IDL, ovvero un linguaggio per definire le interfacce dei servizi.
$\bullet$ Il Portable Object Adapater : si tratta di un sistema per incapsulare gli
oggetti e per facilitare il compito del gestore (nel trasporto, nel supporto

alla QoS, . . . ). È un sistema che viene aggiunto quindi dal supporto!
$\bullet$ L'Interface Repository, che è un sistema simile ad un name server, dove si
possono ritrovare appunto le interfacce dei servizi disponibili.
$\bullet$ Vi sono quindi dei sistemi per richiamare i servizi in maniera statica e
dinamica
$\bullet$ E inne vi sono degli appositi protocolli per la comunicazione ed integrazione fra ORB diversi.
\subsection{L'ORB}
L'ORB è il cuore del sistema, per comunicare si usa sempre l'ORB. Si potrebbe
quindi pensare che è un collo di bottiglia, ma in realtà la comunicazione presente
in CORBA è sempre a {`}grana grossa' (si passano oggetti, servizi pesanti): le
operazioni sono quindi consistenti, e la comunicazione si paga. L'ORB aiuta
fornendo diversi sistemi per facilitare il trasporto della richiesta da parte del
client al server, e la risposta lungo il percorso opposto. Si tratta, quindi, di base
di una comunicazione sempre sincrona.
In particolare, l'ORB gestisce i servitori, preoccupandosi di fare l'allocation
degli oggetti (è lui che deve trovare il server). In particolare, l'ORB permette
che un cliente si leghi ad un servizio, non ad un servitore! CORBA è pensato
infatti per essere da supporto ai sistemi legacy: non interessa chi serve, ma
cosa serve! Fornisce quindi un supporto continuo, e sfrutta tutti i sistemi che
verranno descritti in seguito (POA, interface repository, . . . ).
\subsection{L'IDL}
Il linguaggio per le interfacce è fondamentale per poter definire il contratto che
si vuole realizzare (quindi la descrizione di tutti i servizi da offrire). Questo
contratto sarà quindi incorporato nei vari proxy, ovvero sarà poi compilato nei
linguaggi specici di implementazione!

È mediante IDL che si stabiliscono tutti i collegamenti necessari, e anche
l'ORB lo sfrutta per individuare il servizio (si lavora quindi ad un livello più
63
alto dell'implementazione: l'IDL stesso non è legato a nessun linguaggio di proe
grammazione (deriva dal C++ come specifica, ma è diverso)). Una connessione
statica fa in modo che il proxy del client (lo stub) contenga l'IDL, con cui ci
si interfaccia all'ORB. Questi cerca il servizio descritto dall'IDL, propone la
richiesta in maniera adeguata usando un Object Adapter, e riporta quindi la
risposta al client.
L'IDL è fondamentale per realizzare le connessioni statiche 28 , che sono molto
usate in CORBA (quelle dinamiche, per quanto orano un servizio con maggiore
QoS, sono molto costose). L'IDL definisce (è solo un sistema dichiarativo! un
sistema perché il compilatore scelto realizzi quindi uno stub e skeleton statici
(si può sfruttare per realizzare uno skeleton dinamico on the y mediante un'ino
vocazione dinamica del servizio, aggiungendolo al servitore!). Senza un servizio
di invocazione dinamico, non si può trovare un servizio se non è stato definito
un contratto (ovvero, è stato definito all'interno dell'interface repository!).
IDL resta però sempre e solo una specifica: sarà sfruttato dai compilatori
delle varie implementazioni per creare realmente gli oggetti e i servitori. Infatti,
serve solo come specifica delle operazioni possibili.
Stub e skeleton son necessari per fare le operazioni di marshalling ed unmarshalling. Si ha che si realizza un diverso
proxy (quindi stub e skeleton) per ogni
interfaccia che si definisce.
Da ribadire: è definito un contratto statico fra client e servizio, ma non un
binding statico fra client e servitore!
CORBA non ore altro agli utenti: le interfacce sono gli unici oggetti manipolabili e visibili, e quindi quando si
programma si riferiscono altre interfacce!
L'IDL di CORBA definisce in maniera automatica i metodi di accesso alle
proprietà dell'interfaccia, ed è in grado di fornire delle informazioni sul coma
pletamento corretto o meno dell'operazione. Si possono quindi definire (come
in qualunque altra interfaccia) attributi, operazioni ed eccezioni. Le interfacce
si possono ereditare, anche in maniera multipla, e si possono raggruppare in
moduli logici coerenti.
Di particolare interesse, nei tipi, è la possibilità di definire attributi di tipo
ANY : un attributo di tale tipo può essere sia un valore che un riferimento. Fono
damentalmente funziona come da contenitore, in grado di fornire informazioni
su cosa contiene! Nell'ultima versione di CORBA, 3, si è stabilito che gli
oggetti devono essere passati by-value, in maniera da superare correttamente
l'eterogeneità.
Restano comunque problemi nel realizzare un mapping corretto fra l'IDL e
i vari linguaggi di implementazione! Per esempio, Java non prevede parametri
28 IMPORTANTE: per legame statico si intende dire solo l'interfaccia, che viene quindi
definita in maniera statica. Il binding fra client e servitore non è statico, qui si specifica solo
che i servizi del servitore devono essere specificati in maniera statica
64
di output. Per ovviare a questi problemi, CORBA realizza delle apposite classi dette holder : è un wrapper che permette
di leggere e scrivere in maniera
coerente al linguaggio utilizzato realmente. Si possono anche determinare automaticamente altre classi d'aiuto dette
helper, in maniera da armonizzare i dati
rispetto a CORBA (narrowing) e di fornire diverse funzioni di varia utilità.
\subsection{Gli Object Adapter}
Gli adattatori sono diventati dei componenti quasi più importanti dell'ORB
stesso, visto che sono gli abilitatori del servizio stesso. L'ORB si preoccupa di
crearlo ed abilitarlo, mediante diverse politiche specificabili.
A cosa serve quindi? A superare le disomogeneità e dierenze che vi sono
fra le varie implementazioni degli ORB, dovute appunto all'eterogeneità (per
esempio, riuscendo a far colloquiare linguaggi non tipati con quelli tipati!).
Questo sistema permette di disaccoppiare in maniera corretta l'ORB dall'implementazione del servizio stesso: l'ORB lo
sfrutta, ma ci pensa l'adattatore a
richiamarlo in maniera corretta a seconda della sua implementazione. L'adattatore riesce quindi ad integrare più
componenti scritti in linguaggi diversi! Senza
un adattatore, l'ORB non riuscirebbe a raggiungere il proxy del servitore, senza
avere conoscenza della sua logica implementativa.
Inizialmente, lo standard prevedeva dei BOA, che erano delle implementazioni più semplici. Ora gli adattatori sono in
grado di fornire servizi diversi a
server diversi!
\subsection{L'Interface repository}
Si tratta del luogo dove vengono depositati gli IDL, e quindi da dove si possono
recuperare le informazioni degli unici servizi utilizzabili nell'architettura!
Questo sistema non è utilizzato nella gestione statica dell'architettura, ma
in quella dinamica. Infatti, è ricavando l'IDL dall'IR che si può realizzare un
binding dinamico, e quindi permettere di aggiungere in maniera dinamica dei
servizi ad un servitore/cliente.
Un IR può essere realizzato in diversi modi, tuttavia è previsto che ogni
interfaccia correttamente compilata ed inserita, venga depositata in un'apposita
struttura ad albero, in base ai moduli che la contengono.
\subsection{Protocolli per la comunicazione fra ORB}
Inizialmente, CORBA era stato progettato per scambiare dati binari, in maniera
da ottenere un'ottimizzazione della comunicazione. In seguito si sono introdotte
estensioni come la possibilità di lavorare in Internet fra diversi ORB, o anche di
realizzare comunicazioni fra ORB diversi con protocolli più generalizzati.
Il problema di utilizzare ORB diversi è però la gestione di riferimenti remoti
fra ORB diversi ! Il servizio si sa che non è legato allo specifico servitore, quindi
65
si dovrà trovare un sistema di rendere questi riferimenti globalmente unici (e
qual è il loro tempo di vita?).
\subsection{Cosa è e cosa non è CORBA}
CORBA è un sistema pesante, quindi non ha senso utilizzarlo per progetti
semplici o dove le operazioni son semplici (tipo, settare il valore di una singola
variabile). CORBA infatti sfrutta risorse a grana grossa, che non si muovono!
Il problema di CORBA è infatti che se l'ORB non funziona (o la macchina che
lo contiene non è raggiungibile) tutta l'architettura è oine!
CORBA non è un ambiente di linguaggio, cioè non crea lui gli oggetti. L'ate
tivazione degli oggetti, la loro politica di gestione è tutta a carico degli Object
Adapter, che variano in base all'implementazione! Non vi è quindi uno stane
dard, è il POA a decidere come/se salvare lo stato, . . . Infatti, i servant sono
delle entità passive, a cui il POA porta la richiesta del client e da cui richiede
la risposta corrispondente!
CORBA fondamentalmente richiede la presenza di un riferimento remoto
per poter lavorare: le funzioni base di CORBA sono quindi quelle di gestione e
comunicazione dei riferimenti remoti (trasformazione da/a stringa).
\subsection{Confronto fra Java e CORBA, e sua evoluzione}
I due sistemi sono abbastanza compatibili, proprio perché hanno obiettivi diversi
(Web invece che integrazione di sistemi legacy) e lavorano utilizzando risorse
diverse (a grana ne vs grana grossa).
CORBA ha subito diverse evoluzioni, guidate dalle esigenze delle varie aziende.
Ogni evoluzione ha mantenuto i sistemi e gli oggetti già introdotti, cercando di
estendere i concetti già introdotti (introducendo nuovi linguaggi, aumentando i
servizi, . . . ). Per esempio, CORBA 3 introduce altri sistemi di comunicazione
oltre alla maniera sincrona bloccante.
\subsection{Invocazione statica}
`
E il modello base, ed è anche quello maggiormente utilizzato: permette solo una
comunicazione di tipo sincrona bloccante29 . Si basa sull'utilizzo di due diversi
proxy dal lato del client e del server (stub e skeleton). L'attivazione del servitore
però è dinamica: se non è presente, il POA si preoccupa di attivarlo.
oe
L'invocazione statica bloccante è quella che costa di meno (semantica ate
most-once), però può anche essere molto limitante. Questo perché i client e
i servant devono avere già deciso i proxy in maniera statica, e quindi anche i
servizi! Tutto prima dell'esecuzione!
Inizialmente si era introdotta anche un altra modalità, detta one-way, ma è
deprecata (semantica best-effort, senza possibilità di introdurre QoS).
29 Da ricordare che statico non significa che il client è legato allo specifico servant, ma che il
contratto è deciso prima dell'esecuzione
66
\subsection{Compiti e politiche di un adattatore}
Non è un componente CORBA, ma che dipende dalle varie implementazioni:
il loro compito è quello di dialogare con il servant, il quale si registra presso
l'adattatore corrispondente. In un certo senso, quindi, fanno le veci di un sistema
di nomi, fornendo ai servant anche le risorse dinamiche necessarie.
La semantica degli adattatori è sempre sincrona. Vi sono diverse modalità
d'attivazione iniziali:
$\bullet$ Thread per request: per ogni richiesta l'adattatore30 si preoccupa di creare
un thread, quindi si ha una creazione by-need. Ha quindi un costo abbastanza elevato.
$\bullet$ Pool di thread : I thread sono pre-creati, in maniera da ridurre il costo
d'attivazione (consumano poche risorse se non lavorano). Si deve quindi
prevedere una coda delle richieste, per poter associare ad ogni richiesta un
suo thread.
Si potrebbe realizzare un pool dinamico, in grado quindi di estendere la
sua dimensione in base all'esigenze che si riscontrano durante l'esecuzione.
Il costo è più limitato, ma aumenta il tempo d'attesa.
e u
$\bullet$ Thread per sessione: è un sistema per fornire un maggiore accoppiamene
to. Si trasferisce infatti la responsabilità sul client, permettendogli quina
di di battere sullo stesso servant rispetto alla sessione. Si hanno quindi delle richieste sequenzializzate. Questo
sistema limita ulteriormente il
parallelismo.
$\bullet$ Thread per servant: ogni oggetto viene incapsulato in un singolo servant,
per cui si limita ulteriormente la parallelizzazione. Prima di poter servire
un'ulteriore richiesta, il servant deve concludere la richiesta precedente.
Si hanno dei problemi da considerare nel caso di ORB distribuiti: infatti, si
immagini che si passi un riferimento remoto ad un client da un ORB diverso,
utilizzando come modello il thread per sessione. CORBA dice che è possibile
mantenere le stesse politiche anche se si passa il riferimento all'esterno, ma
in realtà si potrebbe avere che un client diverso utilizzi delle politiche diverse
(potrebbe non essere necessario, per limitare i costi, costringere il client ad utilizzare lo stesso servant). Si hanno
così meno garanzie (non mantengo la stessa
\i{}
politica), ma si ottengono anche dei costi limitati. Si potrebbero anche avere
politiche diverse fra ORB, dovuto tutto ad implementazioni diverse!
Gli adattatori devono quindi garantire certte funzionalità, e nel corso del
tempo è diventato sempre più un componente centrale dell'architettura. Cone
trollano l'esecuzione delle operazione, forniscono un sistema per cui l'ORB non
si preoccupi realmente dell'implementazione, ma per cui può semplicemente
`
portare la richiesta al servant e riportarne indietro la risposta. E il POA a
30 Sono
gli adattatori che devono fornire le politiche!
67
decidere quindi le politiche d'attivazione e il numero di servant necessari per
l'esecuzione!
Inizialmente si aveva i BOA: questi erano legati a poche interfacce, e ad un
unico ambiente di linguaggio. I POA sono un'estensione, aggiungendo funzionalità, potendo interagire con servant dotati
di interfacce e linguaggi diversi: da
qui la denizione di portable! Non abbiamo infatti una chiara indicazione della
sua locazione.
CORBA è stato il primo standard che ha cercato di realizzare un sistema
mw per gli oggetti attivi: un riferimento è quindi in grado di puntare potenziale
mente a più oggetti: un POA riceve il riferimento e deve accedere all'oggetto
desiderato. In generale, meno POA son presenti e meglio è: infatti, le varie
implementazioni forniscono un POA di base, che contiene tutte le politiche di
gestione già definite. Un suo glio non eredita direttamente tutte le politiche, ma
deve essere specificato quali politiche si vogliono attivate: abbiamo quindi in realtà un gestore per i POA, che
stabilisce per ogni POA quante e quali politiche
debba implementare (attiva/disattiva i POA, blocca/scarta le richieste per i
POA). Si potrebbe permettere all'utente di modicare dinamicamente l'active
object map, ma è fatto di rado.
Essendo il responsabile dell'oggetto, il POA è responsabile anche del suo
riferimento. Deve essere in grado quindi di denirlo, identicare gli oggetti in
base all'ID proposto e gestire i servant. In particolare, deve riuscire a distinguere
fra oggetti transienti e quelli persistenti.
Quali sono quindi le politiche d'attivazione per gli oggetti? Possiamo avere:
$\bullet$ Explicit Object Activation: è tutto a carico del client
$\bullet$ Single servant: viene attivato un servant per ogni richiesta, uno solo.
$\bullet$ On demand : sono le politiche per poter scegliere i servant in base alle richieste specificate. Se si tratta di
un singolo metodo non vi è stato, altrimene
ti si tratta di On-demand per una durata indefinita. Una caratteristica
particolare è che le politiche si possono combinare!
\subsection{Binding dinamico}
Avendo a disposizione prima le interfacce, si possono implementare direttamente client e servant. Tuttavia, è mediante
l'invocazione dinamica con cui
si possono aggiungere nuovi servizi durante l'esecuzione. L'alternativa sarebbe
dover spegnere il sistema, e riconfigurare il tutto. La dinamicità si basa sull'uso
dell'IR, e permette di introdurre in CORBA altri modelli di comunicazione oltre
al sincrono bloccante.
Si vuole quindi chiedere o fornire un servizio senza avere il proxy/skeleton è già pronto. E un problema di tutti i mw.
Nel caso del client, si parla di Dinamyc Invocation Interface. Si realizza un oggetto che funziona come il proxy che si
avrebbe avuto in maniera statica. Si ha quindi un oggetto Request, che è uno pseudo-object 31 , che permette di
specificare e una richiesta dinamica. Si tratta quindi di un contenitore fornito dall'ambiente,
per cui un client effettua una invoke, e l'ORB utilizza il servant giusto per
richiamare l'operazione e fornire il risultato. Si potrebbero realizzare tutte le
operazioni dinamiche, ma hanno un costo molto elevato! Tuttavia, vi sono
diversi modi per ottenere il risultato, non solo in maniera sincrona bloccante
(utilizzando la GetAnswer), realizzando quindi delle operazioni maggiormente
disaccoppiate. Si definiscono quindi la:
$\bullet$ Send-deferred per realizzare un sistema sincrono non bloccante. Si utilizza
un sistema basato su un oggetto poll-response.
$\bullet$ Send-one-way invece per avere una comunicazione asincrona.32
Ma come funziona? L'oggetto Request,opportunatamente costruito dal client
(dati, eccezioni, . . . ), passa per l'IR, sfruttandolo proprio come un name server: lo interroga per sapere quale
servant possiede quel servizio (quindi non si
richiede che il servant sia dinamico!). Quindi, l'oggetto Request deve essere particolarizzato in maniera opportuna
dall'utente, in maniera da ottenere l'oggetto
desiderato (ridotta trasparenza!).
CORBA introduce anche la possibilità di estendere dinamicamente i servizi
oerti da un servant (è meno utilizzato). Come fare ciò in realtà non è spiegato
(si potrebbero attivare in maniera dinamica): CORBA si preoccupa di standardizzare le interfacce. Se si avesse un
sistema solo statico, si avrebbe che si
dovrebbe spegnere il sistema e aggiornare staticamente i servant.
Nel modello dinamico (Dinamyc Skeleton Interface) si ha invece che è pree
sente un altro tipo di pseudo-object detto ServerRequest, mediante il quale si può
specificare le operazioni, parametri e contesti da aggiungere! L'idea è che queste
informazioni si debbano agganciare ad un'interfaccia che esiste già. Quindi, ci
si collega all'IR per ottenere un'implementazione già presente, e ci si registra
presso l'IR come sistema in grado di rispondere (il servant si potrà usare sia in
richieste statiche che dinamiche!).
\subsection{Politiche dei riferimenti di CORBA}
CORBA si basa totalmente sull'uso degli ObjRef. CORBA fornisce quindi dei
sistemi/funzionalità di base (per realizzare un primo supporto, da estendere a
seconda della QoS desiderata) per trattare in maniera opportuna i riferimenti
(da string a riferimento o viceversa, funzioni di narrowing, duplicazione/release
dell'oggetto). Gli ObjRef sono opachi e gestibili solo dall'ORB.
La politica però che persegue CORBA è che il riferimento punti al servizio,
non al servitore: tuttavia, questa logica potrebbe non essere quella desiderata dal programmatore (si vorrebbe poter
accedere sempre allo stesso servant
31 Si tratta di facilitatori, connati solo nell'ORB. Non si possono quindi riferire da CORBA,
e non possiedono helper o holder
32 Qual è il problema di lavorare in maniera dinamica? E che si possono avere errori a
`
run-time, che sono più dicili da gestire!
69
sico). Ma quindi questi riferimenti non sono univoci: localmente si potrebbe
anche puntare ad oggetti diversi, realizzando quindi incomprensioni e problemi
(specialmente lato server)!
In CORBA 2 si è quindi introdotta la possibilità di puntare ad uno servant
specifico, introducendo quindi un sistema per definire dei nomi univoci !
\subsection{Interoperabilità fra ORB}
Per poter realizzare le comunicazioni fra ORB diversi, si è dovuto quindi reale
izzare degli ObjRef interoperabili : questi sono gli IOR. Hanno un costo pesante
a livello di implementazione, soprattutto perché una volta stabilito, deve essere
valido in eterno!
Gli IOR facilitano le gestioni degli ORB: si ha che se si registrano client/server
e si vogliano riferire in remoto attraverso una rete di ORB, si definisce uno IOR
sempre valido. In generale adesso gli oggetti in CORBA non sono mai deallocati
automaticamente (potrebbe farlo), ma è tutto a carico dell'utente. Se si utilizza uno IOR, l'oggetto deve essere
sempre presente!
Uno IOR è costituito da una serie di informazioni per renderlo univoco: il nome/indirizzo del nodo su cui risiede, un
timestamp, il nome del POA che lo ha
creato ed altre informazioni. Si realizza quindi in generale uno IOR incapsulando
in maniera opportuna un ObjRef.
Caratteristica importante degli IOR è che non è detto che puntino direte
tamente ad un adattatore, ma possono anche riferire invece un altro repository, ovvero quello degli oggetti
implementati! Si ha quindi la possibilità di
definire un legame indiretto oltre a quello diretto: la differenza sostanziale è che l'oggetto riferito indirettamente è
veramente permanente, mentre riferendolo direttamente si hanno più gradi di libertà, un collegamento maggiormente
lasco. L'ObjId di uno IOR riferisce quindi il servizio sicuramente, ma non è detto l'oggetto.
\subsection{Gli interceptor in CORBA}
Sono definiti come negli altri mw, ovvero è un sistema per poter definire dei come
ponenti che devono essere attivati prima o dopo l'esecuzione di una determinata
richiesta (per introdurre per esempio un sistema di sicurezza). Gli interceptor
son pensati per poter lavorare a diversi livelli (applicativo, trasporto, . . . ).
\subsection{Estensibilità di CORBA}
CORBA è pensato per essere facilmente estensibile. In particolare, CORBA
3 introduce l'uso di componenti, la possibilità di definire QoS, e un supporto
anche ad Internet come ambiente di lavoro.
I componenti non sono altro che degli oggetti inseriti in un apposito contenitore, ed inseriti in un apposito ambiente
d'esecuzione, detto engine. Il servizio
di questi componenti è quindi fornito soltanto nell'ambito del container stesso,
70
sfruttando delle politiche di default definite proprio dal contenitore. Un esempio
di componente di questo genere sono gli EJB.
I componenti sono un fattore trainante per ottenere una base d'utenza sempre più ampia: aumenta la possibilità di
adattare il sistema alle proprie esigenze.
Con CORBA 3 in particolare si è intodotto il sistema dell'Asinchronous
Method Invocation, che è un ulteriore modo per definire una comunicazione
sincrona non bloccante: il cliente non deve attendere quindi per il risultato.
Si ottiene cambiando solo l'interfaccia, per cui si delega un altro oggetto per
attendere la risposta, separando la fase di richiesta da quella callback. Tuttavia,
è necessario che sia il cliente a specificare il codice della callback da eseguire.
Inoltre, si può anche definire che il client recuperi il risultato usando un pollo
object. Nel caso di call-back quindi si utilizza un sistema di re and forget. In
entrambi i casi si lavora solo dal lato del client, per estendere la funzionalità.
Il servant continua a lavorare in maniera sincrona bloccante! Ciò che cambia
oltre a chi richiama il metodo (nel primo caso è l'ORB, nel secondo è il client),
l'interfaccia è leggermente diversa: nel poll i parametri necessari sono di out!
Questo genere di interazione è molto diusa, perché permette di richiedere
contemporaneamente molte {`}istruzioni'.
\subsection{Servizi già presenti in CORBA}
Come già detto, CORBA fornisce diversi servizi già utilizzabili.
Per esempio, per realizzare un servizio di nomi si possono usare il naming
service già presente, oppure addirittura un trading service. La differenza fra
i 2 sistemi è che il name service permette di categorizzare i servizi in base ad
un nome logico (si fornisce il nome logico, si ottiene il riferimento al servizio),
mentre il trading service fornisce un sistema simile a quello delle pagine gialle,
per cui i servizi sono catalogati in base al loro contenuto. In questo caso si
ricerca quindi per chiavi: a default vengono forniti tutti i servizi, ma vi è la
possibilità di ltrare i risultati, per ridurne il numero.33
In entrambi i casi si può pensare (a differenza di Java RMI) di realizzare
dei sistemi federati, ovvero una gerarchia di name o trader server federati, in
maniera di essere coordinati!
Un altro servizio che viene fornito è quello di poter lavorare ad eventi e
notiche: questi infatti non sono strumenti propriamente tipici del modello ad
oggetti, che possono essere sfruttati per realizzare facilmente architetture con
molti clienti e produttori; non è quindi P2P, ma una comunicazione molti a
molti.
CORBA stabilisce delle apposite interfacce, denendo mediante ognuna il
modo di comunicare (push dal produttore, pull dal consumatore, . . . ). Si
definisce un canale come oggetto di delega, per poter inviare i messaggi ai
33 Non
si ricercano ovviamente interfacce, perché presenti sull'IR
71
client, o fare polling ai supplier. Un client si registra quindi al canale, e da
quel momento comincia a ricevere gli eventi.
Di solito, non è noto chi è che genera l'evento, è trasparente al consumatore.
Gli eventi sono non persistenti, senza qualità, e senza possibilità di ltraggio: si
ha che quindi la maniera più comoda per lavorare è il sistema push, e per avere
QoS conviene utilizzare il notication service, che è in grado di fornire qualità
e persistenza!