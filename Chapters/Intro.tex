\chapter{Introduzione}
L'obiettivo del corso è introdurre modelli e tecnologie per poter superare il
modello Client/Server (C/S), in modo
da poter:
\begin{itemize}
 \item Migliorare la gestione
 \item Introdurre un supporto alla \textit{qualità del servizio, QoS}: questo è fondamentale, perché solo se un sistema
 presenta QoS, allora si può pensare ad un sistema di retribuzione (Internet, mail e gli altri servizi
lavorano \textit{best-effort}): vi sono sistemi ``più standard'', importanti per i sistemi molto carichi.
\item Sviluppare sistemi in grado di far fronte anche ad alta mobilità, l'integrazione di sistemi legacy.
\end{itemize}
Si tratta quindi di un corso per poter descrivere le tecnologie di sviluppo possibili per poter creare un
\textit{sistema distribuito}: cosa conviene scegliere, quando usare una tecnologia invece che un'altra? Ha senso cercare
di realizzare una nuova tecnologia, un middleware\footnote{Un middleware è un qualcosa che si interpone fra
l'applicazione e i livelli più bassi: fornisce servizi di aiuto per lo sviluppatore, mette a disposizione meccanismi per
poter implementare \textit{diverse politiche}} proprio? Per fare qualche esempio, non ha senso usare CORBA se il sistema
necessita di un unico nodo, dove si deve interagire con componenti Java, oppure si devono usare \textit{agenti mobili}
dove è necessaria la mobilità! Vi è un particolare interesse per il progetto e testing (soprattutto per l'esecuzione e
il deployment).