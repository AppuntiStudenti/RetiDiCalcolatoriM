\chapter{I Web Services}
L'idea alla base dei Web Services è quella di fornire dei servizi B2B invece dei classici servizi Web. La logica è che
il Web è l'installato più diffuso e facilmente accedibile al mondo, perché non sfruttarlo?
Si ha infatti che molti mw richiedono configurazioni che vanno per esempio contro certi principi delle aziende
(aperture di ulteriori porte verso l'esterno), aggiungendo dei vincoli organizzativi che possono renderne
l'applicazione ancora più complessa.
Con i Web Services si vuole sviluppare un sistema assolutamente standard (basandosi su XML per descrivere i servizi,
per fornire le risposte...), in grado quindi di superare in maniera facile l'eterogeneità (utilizzo di sistemi solo
standard). Gli obiettivi sono quindi quelli di realizzare delle \textit{Service Oriented Architecture} (mappando
l'interfaccia di interazione ad alto livello) dove i servizi devono essere riusabili in alti contesti aziendali,a scarso accoppiamento,senza stato e componibili o delle Enterprise Application Integration, come SAP, ovvero un sistema
in grado di integrare anche sistemi legacy.
Perché allora non usare sempre il Web, perché esistono anche altri mw ancora? Il problema è che il Web ha un costo
molto elevato, per cui potrebbe essere utile per sistemi che richiedono una grana molto grossa.
\section{Struttura generale de WS}
Un WS funziona mediante l'emissione da parte di un client di una richiesta, per cui si ricerca attraverso un apposito
broker (ma è una struttura diversa dall'ORB di CORBA: funziona soprattutto come un directory per i servizi) il
servizio adatto a soddisfare quella determinata richiesta.
L'idea è che quindi vi siano diversi provider in grado di servire i servizi richiesti. Per realizzare ciò, sono stati
realizzati diversi protocolli standard:
\begin{itemize}
 \item \textit{SOAP}: Simple Object Access Protocol, consiste nel protocollo per poter eseguire le richieste e ricevere
 le risposte, ed è ovviamente XML compatibile.
 \item \textit{WSDL}: Web Service Description Language è il linguaggio basato su XML per poter definire dei WS.
 \item \textit{UDDI}: Universal Discovery Distribution Integration: questo invece è il protocollo che specifica come i
 servizi si possano raccogliere in appositi contenitori, realizzando un apposito sistema di nomi.
\end{itemize}
I servizi sono definiti one-shot, il che significa che sono eseguiti in maniera unica, isolata! Per ogni richiesta, si
attiva un solo servizio. Siamo quindi in presenza di un mw senza stato. Questo è sempre dovuto al fatto che si deve
lavorare con dati a grana molto grossa, che auto-contiene l'interazione: non si assume che il provider abbia stato!
Nelle prime implementazioni dei WS non erano previsti meccanismi per la sicurezza, il management e la QoS. Si aveva
quindi, per esempio, solo crittografia a livello applicativo, il che però non era un sistema standard!
Perché si è deciso di utilizzare XML? Permette di specificare in una maniera fortemente strutturata (realizzando dati
che si possono validare formalmente, e utilizzando DTD/XSD si possono anche controllare dal punto di vista
sintattico semantico), potendo quindi esprimere le preferenze in maniera dinamica!
\section{SOAP}
È il protocollo per la comunicazione, in grado quindi di strutturare l'interazione.
Definisce quindi come si può accedere agli oggetti. Presenta un alto overhead perché si deve specificar tutto: siamo
infatti in un'interazione senza stato, per cui si deve specificare URI... Non vi è quindi nessun supporto semantico,
e vi possono essere più interazioni.
SOAP quindi permette di definire operazioni, dati e parametri. Sfrutta HTTP come protocollo per la trasmissione, e XML
per poter serializzare i dati.
In questo modo, si possono definire operazioni remote sfruttando l'XML: un servitore riceve una richiesta standard, e
la sua implementazione non interessa i WS. Un provider dovrà soltanto rispondere rispettando la specifica fornita con
SOAP! SOAP quindi permette di descrivere l'interazione ma non solo: possiamo avere operazioni senza risultati, o in
grado emulare il comportamento C/S\footnote{Al posto delle RPC, si può definire che la comunicazione avvenga in document-style sempre asincrona}, ma possiamo anche quindi definire se si
utilizza GET o POST per esempio. E un protocollo \textit{stateless}, che non fornisce quindi informazioni semantiche
sulla comunicazione. Non è vero che si parla di operazioni leggere e flessibili, ma anche abbastanza pesanti alle volte.
SOAP si basa sull'idea di incapsulare mediante XML le informazioni necessarie alla comunicazione, come se fosse
un'envelope:
\begin{itemize}
 \item Il body contiene le informazioni richieste per la comunicazione, cioè il contenuto vero e proprio del
 messaggio, o le corrispondenti risposte.
 \item L'header specifica altre informazioni utili per la comunicazione (es, per la sicurezza) ed è spesso vuoto.
 \item Si può anche definire una sezione di fault, nel caso che l'operazione non avesse avuto successo.
\end{itemize}
La comunicazione non è necessariamente P2P ma anche molti a molti. Il problema di SOAP è che è altamente costoso (e
poco efficiente), perché siamo costretti a lavorare a livello applicativo.
\section{WSDL}
È il linguaggio per descrivere sia in maniera astratta che in maniera concreta quali servizi un provider può fornire.
Descrive quindi l'interfaccia, indicando le operazioni disponibili presso un certo provider. E una sorta di IDL.
Quindi, WSDL riesce a descrivere in maniera precisa come una richiesta debba essere strutturata, e come una risposta
sarà ritornata. Si può anche specificare dove il servizio risieda, e come lo si può invocare.
Questo sistema, a differenza di altri già visti, impone che il cliente debba conoscere il servitore: viene a sapere in
maniera concreta chi e come realizza l'operazione desiderata. Si passano quindi i dati per copia con i WS. Si ha quindi
che se non si conosce un servizio, si può richiedere il WSDL corrispondente, analizzarlo, e preparare così un'apposita
richiesta SOAP! Il tutto nell'ottica di ottimizzare le operazioni eseguibili da un possibile mw.
WSDL è in grado quindi di definire due parti, una più astratta (in grado di definire i messaggi (insieme di elementi
tipati), le operazioni (o collezione di messaggi), e le interfacce del servizio (insieme di operazioni)) e una più
concreta, a basso livello (in grado di definire i tipi, come il servizio si lega realmente ad un sistema, alle sue
implementazioni concrete, detto binding (che protocollo si usa? HTTP, TCP...), su quali nodi risiede e i service,
ovvero le implementazioni delle interfacce disponibili. La parte astratta è flessibile, generalizzabile e facilmente
estendibile.
WSDL prevede diversi modi di interazione! Non solo più modi sincroni (con attesa di messaggio o sollecitazione da
parte del client) ma anche asincroni (one-way, o notification: cambia da chi esce il messaggio!).
WSDL è un linguaggio molto pesante, ma che garantisce un'alta flessibilità quindi: permette anche di invertire i ruoli
fra Client e Server! Tuttavia, è fondamentale per poter generare in maniera automatica proxy e stub, oppure per realizzare
dei wrapper appositi per integrare facilmente sistemi legacy!
\section{UDDI}
E il servizio di nomi presente nei WS. Si tratta di un ulteriore linguaggio con cui poter descrivere dove risiedono i
vari servizi, e come aggregare fra di loro più servizi. La prima informazione che si necessita è la possibilità di
descrivere le aziende che forniscono i servizi, per permettere quindi ad un generico utente di poter navigare.
Con UDDI si può infatti realizzare facilmente un sistema di pagine bianche (name server normale), pagine gialle
(ovvero un sistema di trading, si memorizza per categorie) o infine per pagine verdi (sono pagine tecniche aggiuntive
sui servizi offerti). Fondamentalmente, UDDI è un ``name server d'alto livello'', globale e condiviso ma non
gerarchizzato. Questo aspetto presenta grossi problemi di coordinazione e di sicurezza fra i diversi gestori dei
servizi!
Gli attori possibili quindi per UDDI sono due: i publisher di un servizio e i loro requester. Vista la struttura degli
UDDI, un requester dovrà navigare singolarmente gli UDDI, perché appunto non vi è ridirezione fra questi! In maniera
particolare, per aggiungere sicurezza si dovrebbe memorizzare e garantire il publisher!
A tale proposito è stato pensato WSIL(Web Services Inspection Language), dove si rovesciano le responsabilità: in questo modo è il provider a dover
garantire sicurezza, e non è più compito di un eventuale server dei servizi verificare l'identità dei provider. Si
tratta quindi di un protocollo alternativo, in grado di realizzare un sistema decentralizzato, leggero e non
focalizzato sul business. In maniera analoga ad UDDI, sfrutta WSDL per descrivere i servizi!
\section{Considerazioni finali sui WS}
I protocolli realizzati dal W3C sono molto apprezzati da diverse organizzazioni, come le banche in Italia. Tuttavia,
i WS presentano un alto overhead rispetto ad altre tecnologie (è possibile che non sia una comunicazione limitata a
due attori, e si lavora sempre a livello applicativo). Attualmente è una tecnologia ancora molto in fase di sviluppo
e studio (nelle prime versione, i protocolli mancavano molto d'espressività: la sicurezza per esempio era fornita
solo a basso livello).
L'evoluzione che si immagina quindi per i WS è quella di far evolvere il sistema, come una composizione e coordinazione
dei servizi, per poter aprire i WS ad una categoria di utenti più ampia. L'idea sarebbe quindi quella che, per
ottenere un nuovo servizio, si compongano quelli precedenti. Per far ciò sono stati introdotti nuovi linguaggi a
flusso (Business Process Execution Language 4WS: si possono mettere in sequenza, parallelo, in alternativa o in join)
per i WS, che lavorano sempre ad un livello molto alto. L'estensione dei WS in questa direzione è indicata come WS*.
E una visione maggiormente orientata ad perseguire logiche di business.
